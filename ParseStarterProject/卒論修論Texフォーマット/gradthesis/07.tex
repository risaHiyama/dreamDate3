\chapter{結論}
\label{chap:result}

本章では本研究の総括を行う。

\section{本研究の総括}
 睡眠中に外的刺激を与えることで夢の中で仮想現実をのような体験をある促すことができた。本研究のために製作したスマートフォンアプリDreamDateを合計15日間7人の被験者に使用してもらった結果、7人中7人が1回以上連想する夢を見ることに成功した。この結果は外的刺激により夢をある程度操作することは可能を示している。\\
 しかしDreamDateを使用しても経験したことのないことは夢でも見れないということが分かった。DreamDateで再生することのできるのは特定の音と強い関連のある想い出だ。実験の結果 6.3 から自ら作曲した歌、演奏した曲、ダンスをした曲などの直接的な記憶にまつわる音楽を流した被験者 5・6・7 は比較的高い確率で関連性のある夢を見た。一方実際 に自分は体験していないが映画などを通して間接的に体験した記憶にまつわる音楽を流した 被験者 3・4 はあまり関連性のある夢をみることはできなかった。よって流す音がユーザー にとって直接的な体験ほど夢に影響を与えやすいと考えられる。また、音を流すタイミングとしては起きる直前のREM睡眠時がユーザーの睡眠を害し難く最適であると考えらえる。\\
 これまでも音や香りの刺激によって夢を操作する取り組みはされてきた。しかしどのくらいの確率で夢が見れるのか、どのような音が適しているのかなどの具体的な説明がされている商品はこれまでになかった。よって本解明できた点は必ずしも多くはないが、若干なりとも寄与できたと思われる。\\
 DreamDateによってユーザは思い出を夢で再生することができるようになり、物理的に会うことのできない人と会話をしたり、過去の思い出でもう一度過ごすことで睡眠をより楽しむことができるなどこれまでにない新しい睡眠のスタイルの実現となる。

\section{今後の展望}
今回の解決すべき命題は、
\begin{itemize}
\item アプリの製作:\\
実験において被験者から自分たちで音楽の登録ができるようにしたいという意見が出た。ここからDreamDateの機能面での再検討及び改善が望まれる。

\item 音による睡眠の悪影響について医療の専門家に確認:\\
睡眠中に起こされてしまったという意見がでた。音による睡眠の悪影響について医療の専門家に確認する必要性がある。

\item 思い出の多様化:\\
2章の明晰夢で体験したい内容について調査したが、実際DreamDateでは特定の音と強く潜在的に思い出のエピソードと結びついている記憶しか夢の中で再生することができなかった。本研究では視覚や聴覚の刺激しか実験しなかったが、睡眠は他にも数多くの刺激と関係性がある。温度、湿度、振動や、体制などによる刺激も考える必要がある。

\item より大人数の実験を行う\\
音楽を流すタイミング、期間、音の種類をについてより細かい実験を行いたい。そのためにもアプリをいち早くAppStoreに登録することでより多くの人たちに使ってもらうことを目指す。

\item 医療の現場においての有効性を考える\\
本研究の分野が多岐にわたって進めば日々失われる記憶の修復ができたり、忘れたくない人をいつまでも覚えていられるようになる可能性がある。認知症や鬱病を抱えている患者に喜びを与えられたり、悪夢に悩まされている人々を救うことができるかもしれない。

\end{itemize}

の4点であった。