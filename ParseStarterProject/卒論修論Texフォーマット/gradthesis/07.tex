\chapter{結論}
\label{chap:result}

本章では本研究の総括を行う。

\section{本研究の総括}
\section{今後の展望}
今回の解決すべき命題は、
\begin{itemize}
\item アプリの製作:\\
実験において被験者から自分たちで音楽の登録ができるようにしたいという意見が出た。ここからDreamDateの機能面での再検討及び改善が望まれる。

\item 音による睡眠の悪影響について医療の専門家に確認:\\
睡眠中に起こされてしまったという意見がでた。音による睡眠の悪影響について医療の専門家に確認する必要性がある。

\item 思い出の多様化:\\
2章の明晰夢で体験したい内容について調査したが、実際DreamDateでは特定の音と強く潜在的に思い出のエピソードと結びついている記憶しか夢の中で再生することができなかった。本研究では視覚や聴覚の刺激しか実験しなかったが、睡眠は他にも数多くの刺激と関係性がある。温度、湿度、振動や、体制などによる刺激も考える必要がある。

\item より大人数の実験を行う\\
音楽を流すタイミング、期間、音の種類をについてより細かい実験を行いたい。そのためにもアプリをいち早くAppStoreに登録することでより多くの人たちに使ってもらうことを目指す。

\item 医療の現場においての有効性を考える\\
本研究の分野が多岐にわたって進めば日々失われる記憶の修復ができたり、忘れたくない人をいつまでも覚えていられるようになる可能性がある。認知症や鬱病を抱えている患者に喜びを与えられたり、悪夢に悩まされている人々を救うことができるかもしれない。

\end{itemize}

の4点であった。