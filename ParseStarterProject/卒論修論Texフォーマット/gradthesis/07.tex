\chapter{結論}
\label{chap:result}

本章では本研究の総括を行う。

\section{本研究の総括}
 外的刺激を与えることで、拡張現実を睡眠中の夢で体験することはできるのか?
 スマートフォンアプリDreamtravelerを使用した実験から、音をREM睡眠中にユーザーにインプットすることで夢を誘発できるというある程度の成果を出せたと考える。以下にその論拠を述べる。
 7人の被験者に合計15日間Dreamtravelerを睡眠中に使用してもらった結果、最後の7日間に関しては60\%の確率で音に連想する夢をみることに成功した。よって外的刺激により夢をある程度操作することは可能を示している。\\
 実験結果から音に関しては、人の声などを交えるとユーザーが起こされることが分かったので、音声でなく音楽などの方が好ましい。また年齢が若いユーザーの方が比較的夢を操作しやすいということがわかった。私生活が結果に影響をもたらしていることも考えられる。The MILD Techniqueのステップにも含まれている睡眠の前に心を落ち着かせながら特定の夢を見ることを念じる行為や夢日記を書き続ける習慣は明晰夢の確率を上げるのに貢献していることが分かった。睡眠時間が少なく、日々ストレスを感じている被験者は音の有無関係なく会社での仕事の夢を見る。日常的に音楽を聞かないユーザーに対しても効果が得られた。具体的にはコーヒーを飲むたびに同じ音楽を聴いてもらった。すると、睡眠中にその音楽を流したときにコーヒーの夢を見ることができたのである。\\
 他の研究でも、音や香りの刺激によって夢を操作する取り組みはされてきた。しかしどのくらいの確率で夢が見れるのか、どのような音楽が適しているのかなどの具体的な説明がされている商品はこれまでになかった。よって本解明できた点は必ずしも多くはないが、若干なりとも寄与できたと思われる。
 Dreamtravelerによってユーザは思い出を夢で再生することができるようになり、物理的に会うことのできない人と会話をしたり、過去の思い出でもう一度過ごすことで睡眠をより楽しむことができるなどこれまでにない新しい睡眠のスタイルの実現となる。

\section{今後の展開}
今回の解決すべき命題は、
\begin{itemize}
\item アプリの製作
\item 音による睡眠の悪影響についてえ医療の専門家に確認
\item 思い出の多様化
\item より大人数の実験を行う
\item 医療の現場においての有効性を考える
\end{itemize}

の4点であった。

\subsection{アプリの製作}
実験において被験者から自分たちで音楽の登録ができるようにしたいという意見が出た。ここからDreamtravelerの機能面での再検討及び改善が望まれる。

\subsection{音による睡眠の悪影響について医療の専門家に確認}
睡眠中に起こされてしまったという意見がでた。音による睡眠の悪影響について医療の専門家に確認する必要性がある。

\subsection{思い出の多様化}
2章の明晰夢で体験したい内容について調査したが、実際Dreamtravelerでは特定の音と強く潜在的に思い出のエピソードと結びついている記憶しか夢の中で再生することができなかった。本研究では視覚や聴覚の刺激しか実験しなかったが、睡眠は他にも数多くの刺激と関係性がある。温度、湿度、振動や、体制などによる刺激も考える必要がある。

\subsection{大人数での実験を行う}
音楽を流すタイミング、期間、音の種類をについてより細かい実験を行いたい。そのためにもアプリをいち早くAppStoreに登録することでより多くの人たちに使ってもらうことを目指す。

\subsection{医療の現場においての有効性を考える}
本研究の分野が多岐にわたって進めば日々失われる記憶の修復ができたり、忘れたくない人をいつまでも覚えていられるようになる可能性がある。認知症や鬱病を抱えている患者に喜びを与えられたり、悪夢に悩まされている人々を救うことができるかもしれない。


