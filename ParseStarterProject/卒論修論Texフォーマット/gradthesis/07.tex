\chapter{結論}
\label{chap:result}

本章では、本研究の総括を行う。本研究では睡眠中に思い出に関連した音を流すことでその音に関連した夢を見ることを促進するシステム、DreamScapeを提案、施策した。明晰夢をスマートフォンアプリによって誘発するためのある程度の成果と、今後の課題や方針を得られたと考える。それらをほんん研究の総括に示す。

\section{本研究の総括}
本研究では睡眠中に思い出に関連した音を流すことでその音に関連した夢を見ることを促進するシステム、DreamScapeを提案、施策した。DreamScapeによってユーザは思い出を夢で再生することができるようになり、拡張現実を体験できるようになれば、物理的に会うことのできない人と会話をしたり、過去の思い出でもう一度過ごすことで睡眠をより楽しむことができるなど、これまでにない新しい睡眠のスタイルの実現となる。
 拡張現実や睡眠に関する調査から睡眠中の夢をコントロールすることにはニーズは充分あると確信し、より多くの人たちが簡単にアクセスできるようにスマートフォンアプリ、DreamScapeを開発をした。携帯に備わっている加速度センサーでREM睡眠(夢を見ているタイミング)を観測し、そのタイミングでユーザー記憶を思い出させる音を流し、ユーザーの夢を刺激、誘導すことを目的としたシステムである。
 4人の被験者に合計20日間DreamScapeを睡眠中に使用してもらった結果、最後の7日間に関しては65\%の確率で音に連想する夢をみることに成功した。明晰夢を実現するために効果的な音楽の種類や音楽を流すタイミングなどを知ることができた。またDreamScapeの導入により、睡眠がより楽しいものになると見込まれた。


\section{今後の展開}
今回の解決すべき命題は、
\begin{itemize}
\item アプリの製作
\item 音による睡眠の悪影響につちえ医療の専門家に確認
\item 実験の制度をあげて大人数での実験を行う
\item 夢を見るために現実を加工することも検討
\item 医療の現場においての有効性を考える
\end{itemize}
の5点であった。

\subsection{アプリの製作}
実験において、被験者からもっと簡単に音楽の登録をできるようにしたいという意見が出た。ここからDreamScapeの機能面での再検討及び改善が望まれる。

\subsection{音による睡眠の悪影響につちえ医療の専門家に確認}
睡眠中に起こされてしまったという意見がでた。音による睡眠の悪影響について医療の専門家に確認する必要性がある。

\subsection{実験の精度をあげて大人数での実験を行う}
実験の精度をあげて音楽を流すタイミング、期間、音の種類をについてより細かい実験を行う。そのためにもアプリをAppStoreに登録することでより多くの人たちに使ってもらう。

\subsection{夢を見るために現実を加工することも検討}
DreamScapeの最大の難点は思い出に関連する音楽を見つけなければならないことだ。人によっては音楽をあまり聞かない人もいる。そこで生活のあり方を変えてみるのだ。例えばコーヒーを飲む時に必ず特定の音楽を聞くことで、その音楽を流すことで夢が操作される可能性が高まる。しかし生活がDreamScapeを使用することで変容していくと予想されるため、それが人によって良い影響であるのかの検討も必要である。

\subsection{医療の現場においての有効性を考える}
本研究の分野が多岐にわたって進めば日々失われる記憶の修復ができたり、忘れたくない人をいつまでも覚えていられるようになる可能性がある。認知症や鬱病を抱えている患者に喜びを与えられたり、悪夢に悩まされている人々を救うことができる。
