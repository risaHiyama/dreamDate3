\chapter{結論}
\label{chap:coding}
%5.1がまだまだ短すぎ。先輩たちの卒論や修論をまずは読んでみること。実験結果の考察を踏まえて、自分が実現しようとしたことがどの程度実現できてどの程度実現できなかったか、を明確に述べること

本章では本研究の総括を行う。
\section{本研究の総括}
 本研究では明晰夢を実現できる新しい手法として、レム睡眠・ノ ンレム睡眠かを観測し起きる直前のレム睡眠中にユーザが望む体験に関連のある音を流すスマー トフォンアプリケーションDreameDate を開発した。合計15日間7人に使用してもらった結果、全員が1回以上連想する夢を見ることに成功した。Head Mounted Display(HMD)では体験できないようなユニーク且つ現実的(リアルに存在する登場人物や空間)な仮想空間を一部の被験者に提供することができた。また高価なデバイスを購入をせずに既存のスマートフォンにアプリケーションをインストールするだけで使用開始できるツールを構築することができた。\\
 しかし被験者の数が少ないため信ぴょう性の高い実験結果を残すことができなかった。また Mnemonic Induction of Lucid Dreams (The MILD Technique)\cite{LaBerge} に比べて負担のかからないユーザー体験を提供することが目的であったが、結果的に夢日記や寝る前にアプリの起動をしなければならないなどユーザーの負担が増えてしまった。加えて睡眠中夢を見ている時間帯を利用したDreamDateでは体験できる仮想現実のコンテンツに限界があるということがわかった。コンテンツの多様化を目指して音声編集システムの開発や音以外の刺激について検討する必要がある。\\
 一方で実験を通して夢に影響を与えやすい音の種類やタイミングを明らかにすることができた。睡眠時に流す音はユーザー にとって印象に残っている体験に関連している音楽であるほど、高い効果が見込まれる。音声は被験者を起こしてしまうので避けたほうが良い。音を流すタイミングとしては起きる直前のREM睡眠時が良い。こうすることでユーザーの睡眠を害しにくくなり、夢の内容を覚えている確率も高まる。\\
 これまでも音や香りの刺激によって夢を操作する取り組みはされてきた。しかしどのくらいの確率で夢が見れるのかなどの具体的な説明がされている商品はこれまでになかった。よって本解明できた点は必ずしも多くはないが、若干なりとも寄与できたと思われる。

\section{今後の展望}

第4章で述べたが、DreamDateには検討されるべき課題が多い。今後の課題を以下に述べる。

\begin{itemize}
\item より正確且つ大規模な実験\\
本研究での予備実験1,2は筆者自身を含めて家族に協力してもらう結果になった。被験者が家族であると信ぴょう性が低いとみなされるので被験者を再び検討して実験をやり直す必要性がある。また睡眠は被験者の寝ている空間、夢を覚えているか否かの体質、その日の行動や体調が実験の結果に大きく影響を与える。そのため被験者の数をできるだけ多く用意する必要があったのだが、最後の実験でも被験者を7人しか集めることができなかった。当初目的としていた信頼性高いデータを集めるためにも、大規模な実験を行うためにAppStoreにDreamDateを登録することを目指す。\\

\item DreamDate機能面の改善\\
DreamDateは実験用に製作したアプリであるため、App Storeで一般のユーザーに公開するためには機能面での再検討及び改善が望まれる。機能面ではまずユーザー自身が音楽の登録をできるようにする。次に睡眠中、節電のためにディスプレーをOFFにする。DreamDateは一晩中加速度によりモニタリングをする必要がある。現時点ではディスプレーがONになっているので、無駄に電力を消費してしまっている。スマートフォンに備わっている光センサーを利用して、ユーザーがスクリーンを伏せたらディスプレーをシャットダウンさせるシステムにする必要がある。またユーザー体験を向上するために、夢日記機能の手間を減らす必要がある。\\

\item DreamDateユーザー体験の改善\\
現時点ではユーザーは起床後すぐにテキストを入力する仕組みになっているが、音声録音にすることで負担を減らすべきである。また寝る前に他のアプリを使用できない、一晩中充電をしなければならない、睡眠中に音によって起こされてしまうなどの問題を解決してユーザー体験の改善をする必要がある。最後に夢というのはプライベートな情報なので夢日記などのプライバシー強化のための仕組みも必要になるだろう。\\

\item コンテンツの多様化\\
2章の明晰夢で体験したい内容について調査をして、LOVE タイプ、癒しタイプ、元気欲しいタイプ、アドベンチャータイプ、ストーリータイプや、ビジネスタイプなど様々な要望があるのに対して、DreamDateでは特定の音と強く潜在的に思い出のエピソードと結びついている記憶しか夢で再現することができなかった。音声はユーザーを起こしてしまうということで今回は断念せざる得なかったが、音声と音楽をリミックスするなどをして、音声を自動に編集するシステム検討する必要がある。他にも視覚や聴覚の以外にも温度、湿度、振動や体制などによる刺激も試す必要がある。最後に4章でも条件反射の可能性について述べたが、生活の中で特定の行動をするときに音楽を聴く習慣をつけることでどれだけ夢に影響を与えることができるかを実験を通して証明する必要がある。\\

\item 長期的利用が被験者に及ぼす影響を調べる\\
長期的な実験における被験者の生活への影響を考慮するを必要があるだろう。今回は長くて15日間の実験となったが、長期的使用によって睡眠に支障が起きないかを調べる必要がある。睡眠は生物が体を休めるために必要不可欠な行為である。今回DreamDateを使った被験者が睡眠中に起こされてしまったという意見が多数でた。音による睡眠の悪影響について医療の専門家に確認する必要性がある。また明晰夢は経験を重ねるほど、成功率が上がると言われている\cite{LaBerge}。2ヶ月程実験を続けて効果に変化が出るか実験をする必要がある。一方でユーザーが自由自在に夢を操作できるようになった場合、ユーザーがどのような心境になるのかを調べる必要性がある。被験者Cは予備実験を通して夢から覚めて、現実に戻った際に切ない気持ちになってしまった。明晰夢が原因で現実を受け止められれなくなり、結果的にユーザーが悲しむ事態に陥ってしまわないか確かめる必要がある。\\

\item 医療の現場においての有効性を考える\\
本研究の内容が医療の分野で活用されれば、日々失われる記憶の修復ができたり、認知症患者が忘れたくない人や事柄をいつまでも覚えていられるようになる可能性がある。近年ストレスから鬱病や不眠症の悩み抱える社会人が増えているが、DreamDateが夢の中で喜びを与えることで多くの人々を救うことができるかもしれない。\\
\end{itemize}