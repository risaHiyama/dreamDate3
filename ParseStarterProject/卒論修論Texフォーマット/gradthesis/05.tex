\chapter{結論}
\label{chap:coding}

本章では本研究の総括を行う。
\section{本研究の総括}
 本研究では明晰夢を実現できる新しい手法として、レム睡眠・ノ ンレム睡眠かを観測し起きる直前のレム睡眠中にユーザが望む体験に関連のある音を流すiOSスマー トフォンアプリケーションDreameDateを開発した。睡眠中夢を見ている際に記憶の整理をしている脳の習性を利用して、DreamDateでは思い出と関連性のある音を使った。

本研究では以下の点を検証することを目的にしていたが
\begin{itemize}
\item 睡眠中のユーザを音で刺激することでどのくらいの確率で明晰夢を促せるのか
\item DreamDateによってMILDに比べてより負担のかからないユーザ体験を提供できるのか
\item DreamDateによってHMDよりもオリジナルで現実世界に近いコンテンツを体験できるか
\end{itemize}

合計15日間のDreamDate7人の被験者が使用した結果、全員 1 回以上関連する夢を見ることができた。予備実験1と本実験の結果から10人中 8人が、音で刺激を与えた夜の方が与えなかった夜に対して明晰夢を見る確率が高かったことから、音が夢への影響を持っていることが示唆される。DreamDateは高価なデバイスを購入をせずに既存のスマートフォンにアプリケーションをインストールするだけで使用開始できるツールを構築することができる。DreamDateは音や画像の登録、夢日記の記入などの作業を必要とするが、Mnemonic Induction of Lucid Dreams (The MILD Technique)に比べて負担がかからない\cite{LaBerge} 。DreamDateはHead Mounted Display(HMD)によるシュミレーションでは体験できないようなユニーク且つ現実的(ユーザの生活の中で実在する人物や空間、思い出に由来した体験)な仮想の体験を一部の被験者に提供できた。その効果に個人差が見られたが、DreamDateの夢制御システムの有効性が証明された。さらにDreamDateの有効性を上げる要素として下記が挙げられる。

\begin{itemize}
\item 睡眠時に流す音はユーザの思い出に直接的な音を選ぶ
\item 起きる直前のレム睡眠時に音を流す
\item DreamDateをより長く使用し続ける
\end{itemize}

DreamDateはユーザ自身の思い出と関連性のある夢を見たいというときは効果を表したが、今後はコンテンツの多様化を目指して音声編集システムの開発や音以外の刺激について検討する必要がある。本研究の実験は全体的に被験者の数が少ないため説得力のあるデータを収集できたとはいえないが、実験を通して明晰夢制御システムの開発にあたって意識するべき様々なポイントを発見できた。

\section{今後の展望}
DreamDateにはまだ改善するべき課題を多く残している。今後の課題を以下に述べる。

\subsection{スマートフォンアプリの課題}
DreamDateの機能改善\\
DreamDateは実験用に製作したアプリであるため、App Storeで一般のユーザに公開するためには機能面での再検討及び改善が望まれる。機能面ではまずユーザ自身が音楽の登録をできるようにする。次に睡眠中、節電のためにディスプレーをOFFにする。DreamDateは一晩中加速度によりモニタリングをする必要がある。現時点ではディスプレーがONになっているので、無駄に電力を消費してしまっている。スマートフォンに備わっている光センサーを利用して、ユーザがスクリーンを伏せたらディスプレーをシャットダウンさせるシステムにする必要がある。またユーザ体験を向上するために、夢日記機能の手間を減らす必要がある。\\

DreamDateユーザ体験の改善\\
現時点ではユーザは起床後すぐにテキストを入力する仕組みになっているが、音声録音にすることで負担を減らすべきである。また寝る前に他のアプリを使用できない、一晩中充電をしなければならない、睡眠中に音によって起こされてしまうなどの問題を解決してユーザ体験の改善をする必要がある。最後に夢というのはプライベートな情報なので夢日記などのプライバシー強化のための仕組みも必要になるだろう。\\

\subsection{明晰夢を制御することの課題}
より正確且つ大規模な実験\\
本研究での予備実験1,2は筆者自身を含めて家族に協力してもらう結果になった。被験者が家族であると信ぴょう性が低いとみなされるので被験者を再び検討して実験をやり直す必要性がある。また睡眠は被験者の寝ている空間、夢を覚えているか否かの体質、その日の行動や体調が実験の結果に大きく影響を与える。そのため被験者の数をできるだけ多く用意する必要があったのだが、最後の実験でも被験者を7人しか集めることができなかった。当初目的としていた信頼性高いデータを集めるために、明晰夢に意欲的な10人以上の被験者に50日以上の実験に関わってもらう。同時にAppStoreにDreamDateを登録しより多くの人に使ってもらうことを目指す。\\

コンテンツの多様化\\
2章の明晰夢で体験したい内容について調査をして、LOVE タイプ、癒しタイプ、元気欲しいタイプ、アドベンチャータイプ、ストーリータイプや、ビジネスタイプなど様々な要望があるのに対してDreamDateでは特定の音と強く潜在的に思い出のエピソードと結びついている記憶しか夢で再現することができなかった。予備実験2で「音声」「曲」「自然音」の3つがそれぞれ与える影響を比較した際に交際相手に名前を呼ばれたり、語りかけられている音声を流した夜は音刺激により起こされて睡眠を害されてしまった。その後睡眠中のインプットは曲や波や森林などの自然音が好ましいと判断しその後のDreamDateの実験を重ねたが、音声でも語りかけ口調ではなく説明口調や歌声で再度実験を行うべきである。また音声と音楽のリミックスなどを自動に制作する音楽編集システム検討したり、視覚や聴覚の以外にも温度、湿度、振動や体制などによる刺激も試す必要がある。最後に4章でも条件反射の可能性について述べたが、生活の中で特定の行動をするときに音楽を聴く習慣をつけることでどれだけ夢に影響を与えることができるかをより多くの被験者を集めて実験行うべきである。\\

長期的利用が被験者に及ぼす影響を調べる\\
長期的な実験における被験者の生活への影響を考慮するを必要があるだろう。今回は長くて15日間の実験となったが、長期的使用によって睡眠に支障が起きないかを調べる必要がある。睡眠は生物が体を休めるために必要不可欠な行為である。今回DreamDateを使った被験者が睡眠中に起こされてしまったという意見が多数でた。音による睡眠の悪影響について医療の専門家に確認する必要性がある。また明晰夢は経験を重ねるほど、成功率が上がると言われている\cite{LaBerge}。2ヶ月程実験を続けて効果に変化が出るか実験をする必要がある。一方でユーザが自由自在に夢を操作できるようになった場合、ユーザがどのような心境になるのかを調べる必要性がある。被験者Cは予備実験を通して夢から覚めて、現実に戻った際に切ない気持ちになってしまった。明晰夢が原因で現実を受け止められれなくなり、結果的にユーザが悲しむ事態に陥ってしまわないか確かめる必要がある。\\

医療の現場においての有効性を考える\\
本研究の内容が医療の分野で活用されれば、日々失われる記憶の修復ができたり、認知症患者が忘れたくない人や事柄をいつまでも覚えていられるようになる可能性がある。近年ストレスから鬱病や不眠症の悩み抱える社会人が増えているが、DreamDateが夢の中で喜びを与えることで多くの人々を救うことができるかもしれない。\\