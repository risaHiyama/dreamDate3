\chapter{実験方法とその結果}
\label{chap:ledoxea}

本章では、開発したiOSアプリ TokimekiDream の実験方法、内容、結果の詳細について説明する。

\section{実験方法}
予備実験:下記の被験者にまずどのような音楽が異境を与えやすいのか、どのタイミングで流すと影響が出やすいのかを調べるために実験を重ねた。

\subsection{実験結果}
夢は音に影響を与える。音を流すタイミングは起床直前のREM睡眠が一番効果的であった。そして音量は1〜3がちょうど良い。それ以上だと起きてしまう。iPhoneの位置は枕の横の方が良い。枕の下だとiPhoneが熱を発し流だけではなく、脳に音楽が響き渡り、目が覚めやすい。最後に音の種類は声、特に語りかけ口調の音声だとユーザーが起きてしまう確率が高かった。また音声はユーザーの思い出に由来している音楽の方が、影響を与えやすいということがわかった。これらの実験結果を踏まえて本格的な実験をした。それについて次に述べる。

\subsection{インタビュー}
被験者の4人は流す音声を決めてもらうために、インタビューをしてどのような趣味を持っていて、どのような思い出の夢を見たいのかを質問した。

\subsection{実験}
また、夢日記を実験の1週間前から記録してもらった。人は見た夢をすぐ忘れてしまう習性があるため、夢日記を夢を記憶できる体質になってもらう。次に
DreamScapeを20日間使い続けてもらい、REM睡眠中に流れる音声が夢に影響を与えることに成功するかいなかを実験した。実験は音楽なしと音楽ありを交互にして、時には寝る前に思い出に関する画像をみたり、思い出について10分間考えならが寝る意識をしてもらった。

\section{実験結果}
音楽なしのときは・・・(まだ実験中)
音声ありのとき・・・(まだ実験中)
音声・寝る前に画像をみる・寝る前の瞑想ありのとき・・・(まだ実験中)
継続的に使い続けるほど利用に対する希望するか否か(まだ実験中)
