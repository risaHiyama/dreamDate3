\chapter{実験方法とその結果}
\label{chap:ledoxea}

本章では、開発したスマートフォンアプリMemoryDreamの実験方法(調査対象・観察方法)と実験結果のについて説明する。

\subsection{調査対象}
20歳〜50歳の男女5人に参加してもらった。5人の被験者は皆明晰夢に興味がある人。一人は普段から経験している人。一人は一度もしたことない人。音にどれくらい敏感かも述べる。比較的一定の睡眠活動をしている人を対象にした。

被験者1:
\begin{itemize}
\item 国籍:インドネシア人
\item 性別:男性
\item 家族構成:2次の父・既婚者
\item 年齢:30代後半
\item 明晰夢の経験:5回ほど経験している
\item 夢日記を行ったか否か:5日間ほど行った
\item 思い出に由来する音楽:イスラームの聖典「コーラン」
\item 趣味:ブラジリアン柔術
\end{itemize}

被験者2:
\begin{itemize}
\item 国籍:日本人
\item 性別:女性
\item 家族構成:2次の父・既婚者
\item 年齢:40代後半
\item 明晰夢の経験:経験したことなし
\item 夢日記を行ったか否か:10日間ほど行った
\item 見たい夢:沖縄での旅行
\end{itemize}

被験者3:
\begin{itemize}
\item 国籍:日本人
\item 性別:男性
\item 家族構成:2次の父・既婚者
\item 年齢:50代前半
\item 明晰夢の経験:経験したことなし
\item 夢日記を行ったか否か:10日間ほど行った
\item 見たい夢:007映画で出演者になりたい
\end{itemize}

被験者4:
\begin{itemize}
\item 国籍:アメリカ人
\item 性別:男性
\item 家族構成:一人っ子
\item 年齢:20代前半
\item 明晰夢の経験:何度か経験したことがある
\item 夢日記を行ったか否か:5日間ほど行った
\item 見たい夢:アイスランドでの旅行・ジブリの映画
\end{itemize}

被験者5:
\begin{itemize}
\item 国籍:アメリカ人
\item 性別:男性
\item 家族構成:兄弟2人、弟
\item 年齢:20代前半
\item 明晰夢の経験:何度か経験したことがある
\item 夢日記を行ったか否か:5日間ほど行った
\item 見たい夢:アイスランドでの旅行・ジブリの映画
\end{itemize}

被験者6:
\begin{itemize}
\item 国籍:日本人
\item 性別:女性
\item 家族構成:妹
\item 年齢:20代後半
\item 明晰夢の経験:経験したことない
\item 夢日記を行ったか否か:5日間ほど行った
\item 見たい夢:沖縄旅行
\end{itemize}

\subsection{実験方法}
明晰夢の回数と外的刺激の関係性\\
 外的刺激が夢に影響を与えるか否かを調べるために14日間の実験に参加してもらった。実験を通して何も音楽を流さない場合と流した場合の結果の違いを分析した。20日間という実験記録を集めたのは、睡眠に関する実験は体調、その日の活動内容や、被験者の心境によって左右されがちでデーターが変動しやすいためである。
 実験を開始する7日間前から、被験者には夢日記を書いてもらった。そうすることで夢を記憶できる体質になってもらうことを目指した。またスマートフォンは充電をした状態で枕の横に置いてもらうことで、音声が脳に届く状態にしてもらった。

 実験は音楽なしと音楽ありを交互にして、時には寝る前に思い出に関する画像をみたり、思い出について10分間考えならが寝る意識をしてもらった。詳しい事件内容は以下に述べる。

実験シュケジュール

\begin{itemize}
\item 1日目
\item 2日目
\end{itemize}
 
\subsection{観察の方法}
アプリに見た夢の内容を記録してもらった。
明晰夢を見た回数とその時与えていた外的刺激の関係性について調べた。

\section{実験結果}
\subsection{データの提示}

\subsection{解析結果の説明}
音楽なしのときは・・・(まだ実験中)
音声ありのとき・・・(まだ実験中)
音声・寝る前に画像をみる・寝る前の瞑想ありのとき・・・(まだ実験中)
継続的に使い続けるほど利用に対する希望するか否か(まだ実験中)

\subsection{結果のまとめ}
 *解釈は述べじゃダメ。例)示唆しているとか〜
 夢は音に多少影響を与える。年齢の高い人ほど結果が出やすい。また音だけでなく、画像と瞑想をさせた時の方がさらに夢を見やすい。