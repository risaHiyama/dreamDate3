% 独自のコマンド

% ■ アブストラクト
%	\begin{jabstract} 〜 \end{jabstract}	:日本語のアブストラクト
%	\begin{eabstract} 〜 \end{eabstract}	:英語のアブストラクト

% ■ 謝辞
%	\begin{acknowledgment} 〜 \end{acknowledgment}

% ■ 文献リスト
%	\begin{bib}[100] 〜 \end{bib}


\newif\ifjapanese

\japanesetrue	% 論文全体を日本語で書く(英語で書くならコメントアウト)

\ifjapanese
	\documentclass[11pt]{jreport}
	\renewcommand{\bibname}{参考文献}
	\newcommand{\acknowledgmentname}{謝辞}
\else
	\documentclass[11pt]{report}
	\newcommand{\acknowledgmentname}{Acknowledgment}
\fi
\usepackage[top=30truemm,bottom=30truemm,left=25truemm,right=25truemm]{geometry}
\usepackage{ascmac}
\usepackage{graphicx}
\usepackage{multirow}
\usepackage{ylab_thesis}
\DeclareFontShape{JY1}{mc}{m}{it}{<5> <6> <7> <8> <9> <10> sgen*min
    <10.95><12><14.4><17.28><20.74><24.88> min10 <-> min10}{}
\DeclareFontShape{JT1}{mc}{m}{it}{<5> <6> <7> <8> <9> <10> sgen*tmin
    <10.95><12><14.4><17.28><20.74><24.88> tmin10 <-> tmin10}{}
\usepackage{times}
\usepackage[stable]{footmisc}
\usepackage{otf}
\usepackage{epsf}

%\bindermode	% バインダ用余白設定

% 日本語情報(必要なら)
\jclass	{卒業論文}							% 論文種別
\jtitle		{睡眠中の夢は外的刺激により操作できるのか?\\スマートフォンアプリDreamDateの\\提案と検証}			% タイトル。改行する場合は\\を入れる
%明晰夢を実現することで拡張現実を実現することは可能なのか?
\juniv		{慶應義塾大学}						% 大学名
\jfaculty	{環境情報学部環境情報学科}				% 学部、学科
\jauthor	{樋山 理紗}						% 著者 \CID{7808}でも同字が出る
\jnumber	{71247475}						% 学籍番号
\jadvisor	{中西 泰人}{教授}					% 指導教官、形式は『{名前}{肩書}』
\jchief	{}{}					% 学科長名、形式は『{名前}{肩書}』
\jhyear	{28}								% 平成○年
\jhyeared {28}								% 平成○年「度」
\jsyear	{2016}						% 西暦○年
\jsyeared	{2016}							% 西暦○年「度」
\jkeyword	{Lucid Dreaming, iOS, Sleep Monitoring}			% 論文のキーワード

% 英語情報(必要なら)
\eclass	{Graduation Thesis}					% 論文種別
\etitle		{DreamDate iOS App to \\ dream about your desired memory}	% タイトル。改行する場合は\\を入れる
\euniv	{Keio University}						% 大学名
\efaculty	{Bachelor of Arts in Environment and Information Studies}	% 学部、学科
\eauthor	{Risa HIAYAMA}					% 著者
\enumber	{71247475}						% 学籍番号
\eadvisor	{Professor}{Yasuto NAKANISHI}				% 指導教官、形式は『{肩書}{名前}』
\echief	{}{}				% 学科主任名、形式は『{肩書}{名前}』
\eyear	{2016}							% 西暦○年
\ekeyword	{Lucid Dreaming, iOS, Sleep Monitoring}		% 論文のキーワード

\begin{document}

\jmaketitle		% 表紙(日本語)、不要ならコメントアウト
\emaketitle		% 表紙(英語\ref{chap:latex})、不要ならコメントアウト

% ■ 概要の出力 ■
%		begin{jabstract}〜end{jabstract}	:日本語の概要
%		begin{eabstract}〜end{eabstract}	:英語の概要
%		※ 不要ならばコマンドごと消せば出力されない。

% 日本語の概要
\begin{jabstract}
 本研究では仮想現実の新しいアプローチを見出すことを目的とし、睡眠中の夢を音で操作することで仮想現実を体験することができるのか否かについて調べた。\\
 近年Virtual Reality(VR)に関する研究・開発が進んでいる。しかしVRコンテンツの制作には多額の金銭的・時間的なコストがかかりバリエーションに限界があるとされているため、仮想現実を体験するための新しい取り組みが求められる。そこで本研究では明晰夢(睡眠中にみる夢のうち、自分で夢を見ていることを自覚していながら見ている夢のこと)に注目した。明晰夢の体験者は度々夢の方向性を変化させることができると知られている。そして夢の中で現実味があり且つユニークな体験をすることが可能なのだ。しかし明晰夢を試みるには特殊な訓練が必要とされている。\\
 そこで本研究では特別な訓練をしなくても明晰夢を実現することができる新しい方法を検証した。具体的にはDreamTravelerというスマートフォンアプリケーションを開発し、REM睡眠中(夢を見ている時)に音を流し、その音が夢に影響を与えるのか否かについて調べた。\\ 
 DreamTravelerに睡眠の深さを観測する機能、音楽再生機能、夢日記機能を加えた後に、合計7人の被験者に15日間の実験を行った。結果、最後の7日間に関しては60\%の確率で音に連想する夢をみることに成功した。この結果から、音が夢に影響を及ぼすことができるということと、ユーザーにとって音が記憶と関連性が高ければ高いほど夢に影響が高いということが示唆された。

\end{jabstract}

% 英語の概要
\begin{eabstract}
	Virtual Reality became a great trend, however it takes many work to program VR contents. Therefore alternate ways of experiencing extended reality are in huge need. People actually experience extended reality while asleep, during dreams. Therefore I focused on the idea of Lucid Dreaming. Lucid Dreaming is when people notice that they are in their dream, and have control over their dream. Dreamtraveler is an app to direct dreams by playing music to effect their dream. I was able to prove that certain sounds have effect on people's dream. Through out this paper, I will explain the details about the system, the effect which Dreamtraveler had on the target users.
\end{eabstract}	% アブストラクト。要独自コマンド、include先参照のこと

\tableofcontents	% 目次
 \listoffigures		% 図目次、不要ならコメントアウト
% \listoftables		% 表目次、不要ならコメントアウト

\pagenumbering{arabic}

\chapter{序論}
\label{chap:introduction}

本章では本研究の背景、それを踏まえた上での研究の目的、そして文書の構成について述べる。

\section{本研究の動機}
 昨今ではVirtual Realityが開発が進んでいる\cite{vrtrendShiny}。Virtual Realityはコンピューターによりシュミレーションされた拡張現実がユーザの動きにインタラクティブに反応することをいう。図1は年代別に制作されたVirtual Realityの数を示している\cite{vrtrendSamuel}。\\

\begin{figure}[htbp]
\begin{center}
\includegraphics[width=15cm]{eps/vrTrends.eps}
\caption{Virtual Reality トレンド}
\label{Virtual Reality トレンド}
\end{center}
\end{figure}

 注目を浴びているのは現実世界の限界を越えてた体験をしたいというニーズがあるためである。 そのニーズの中には物理的壁を越えて遠くにいる人と会いたい、時間的壁を越えて過去に旅行をした時に思い出を再び体験したい、自ら好きな映画の登場人物になって刺激を感じたいなどというニーズが含まれる。Virtual Realityは拡張現実をより現実的に感じさせるために、映像の見せ方、インタラクションの行われ方、音の聞こえ方を正確に行わなければならいため開発に労力と時間がかかる\cite{vrtrendShiny}。\\
 しかし実は人は毎晩睡眠中に拡張現実を体験しているのだ。辞書 『大辞泉 第二版』によると、夢とは「睡眠中にあたかも現実の経験であるかのように感じる一連の観念や心像のこと」\cite{dream}と書かれている。中でも明晰夢は1867 年に Saint Denys により研究が行われて以来\cite{saintDenys}、心理学者や哲学者の間で研究が進められてきた。明晰夢とは睡眠中にみる夢のうち、自分で夢であると自覚しながら見ている夢のことである。経験者は夢の状況を自分の思い通りに変化させられる、言い換えれば拡張現実を体験することができると語られてきた。スタンフォード大学博士の Stephen LaBergeは1987 年に The Lucidity Institute を設立し、明晰夢を見るためのステッ プを Mnemonic Induction of Lucid Dreams (The MILD Technique) で紹介した\cite{LaBerge}。しかしThe MILD Techniqueを習得するには特殊な訓練が必要となり、労力が必要で誰もが気軽に始められるものとは言い難い。そこで誰もが比較的簡単に明晰夢をみるとで、拡張現実を体験できるツールの開発が望ましい。

\section{本研究の目的}
 本研究ではVirtual Realityに変わって明晰夢に着目する。そして、寝る前にスマートフォンのアプリを起動するだけで睡眠中に拡張現実を体験できる、DreamScapeの提案と試作をする。具体的にはユーザーの思い出に関連した音声や画像が持つ明晰夢の誘発可能性について証明する。誰もが簡単に夢を操作できる方法を見つけることができれば、物理的に会うことのできない人と会話をしたり、授業で学んだ内容を復習したり、過去の思い出でもう一度過ごすことで睡眠をより楽しむことができるなど、これまでにない新しい睡眠のスタイルの実現となる。
 人生の1/3を過ごす睡眠時間をより有効的に使うことができれば人類の発展に大いに繋がる可能性が高い。睡眠の質の向上を目的として、モニタリング機能を備えたデバイスはiSleep\cite{iSleep}やbeddit\cite{beddit}などこれまでに多くの研究が行われてきた。しかし明晰夢を促進するための研究はまだ少ない。タカラトミーが開発した夢見工房\cite{takaratomi}やDreamON\cite{dreamOn}をはじめとする各種スマートフォンアプリが提案されているが、どれも商業目的のものが多く、信頼性の高い実験データを公開していないため有効性については大いに疑問が残る。そこでDreamScapeは多数の開発者が実現しようとしたことに挑戦し、詳細な実験結果を残したい。

%昨今ではVirtual Reality( VR)が活発に研究されるようになった。 VR とはコンピューターによりシュミレーションされた拡張現実を見せて、ユーザーがインタラクションできるようにすることである。VRの中でもテレイグジスタンスという分野では、各地にあるものや人ががあたかも近くにいるかのように感じながら、操作などをリアルタイムに行う環境を構築する技術および体型のことで、慶應義塾大学 大学院メディアデザイン研究科の舘\UTF{66B2}教授によって1984年に初めて紹介されている。
%一方明晰夢は1867年にSaint Denysにより研究が行われて以来、心理学者や哲学者の間で研究が進められてきた。本研究の中では、明晰夢という定義としてPaul Thoelyの記した定義を流用する。引用Paul Thoelyは以下は明晰夢を以下の条件全てが一致したときにのみ、明晰夢と断定することとしている。(1)夢を見ているということを自覚する。(2)夢において自覚的に判断を行う。(3)起床後も夢に関する記憶がある。(4)夢においての自分の存在を理解する(5)夢の中で自分が置かれている環境を理解する(6)夢の意味を理解する(7)集中度を自覚する。明晰夢の経験者はしばしば、夢の状況を自分の思い通りに変化させられると語っている。明晰夢は娯楽やセラピーにおいて有効な手段として提唱されている。スタンフォード大学博士のStephen LaBergeは1987年にThe Lucidity Instituteを設立して、明晰夢を見るためのステップをMnemonic Induction of Lucid Dreams (The MILD Technique)で紹介した。明晰夢はVRに比べて、コストが低いということと、オリジナルな拡張現実を体験出来るという点で優位に立つが、しかしこのスキルを獲得するには時間と労力が必要とされてきた。
%当研究ではこういった問題点を改良するため、個人の思い出と直結した音声をREM睡眠中のユーザに聞かせることで、仮想空間を体験させることを目的とする。本研究ではMemoryDreamというスマートフォンアプリケーションを使い、音声を流すことで夢をある程度誘発することができるのか否かを探る。

%2007年、iOS、Android OSの両オペレーティング・システムを搭載したタブレット端末やスマートフォン端末が発表された。
%これら端末はスペック的にそれほど高くないものの、タッチパネルの搭載による直感的な操作、携帯性の高さ、Wi-fi接続によるインターネット接続が可能といった多数のメリットを兼ね備えており、欧米を中心に今日まで爆発的に普及してきている。\footnote{株式会社シード・プランニングの行った2012年7月の市場調査\cite{smartphoneresearch}によると、日本でのスマートフォン普及率は40\%前後と先進国の中ではやや低調である。元々高品質な携帯電話が普及しており、プラットフォームとしても盤石であったことが要因であると考えられている。}\\
%一方、e-learningとは、パーソナルコンピュータなどの情報機器を用いて行う学習のことである。1990年代後半からのPCの普及と共に様々な分野で用いられるようになり、現在ではe-learningのコンテンツ共有を目的とした規格\cite{scorm}や大学設置基準に基づく文部科学省告示の中にe-learningに関する項目が記述される\cite{monkasho}など、制度や規格も整備されたものとなっている。\\
%だがe-learningコンテンツを提供するサイトは、その多くがタブレット端末、スマートフォン端末が発表されるより前に製作されたものである。現状、スマートフォンやタブレットからe-learningコンテンツに対してダイレクトにアクセスするためには、まずはうまくコンテンツだけがヒットするような副次的な検索キーワードを考えて、PC用のサイトから小さなボタンをタップし、コンテンツをダウンロードし、更にテキストや動画で別々の検索エンジンを使わなければならない、といったようにかなりの労力を要する。\footnote{iOSについては、e-learning用にユーザーインターフェースが最適化されたiTunesU\cite{itunesu}が存在するが、これはiTunesStore内にあるコンテンツのみを対象としており、WWW上に存在するコンテンツをすべて検索対象とすることはできない。}。\\
%当研究では、こういった問題点を改良するため、スマートフォン端末・タブレット端末上でe-learningコンテンツを簡単に検索し、自在かつ直感的に閲覧、ダウンロードできるアプリの開発を行う。

\section{本文書の構成}
この第\ref{chap:introduction}章では、本論文を書くに至った動機とその構成を説明している。第\ref{chap:webapi}章では本研究の背景を、ユーザーのデザイン要件に関する事前調査・分析と、拡張現実の普及との二点に分けて述べる。第\ref{chap:search}章では睡眠観測や明晰夢促進という目線での先行研究や開発事例を述べたのちに、複数の 観点からDreamSacpeとの比較を行い新しい解決方法について提起する。第\ref{chap:visualize}章ではDreamSacpeがiOSアプリで睡眠中に音を流すという形に至った背景を述べる。第\ref{chap:coding}章では本研究で試作した MemoryDreamについて、その概要、利用方法、システムの概要、実装手法について述べる。第\ref{chap:ledoxea}章では実験方法と、目的に対するMemoryDreamの有効性について試用実験から評価 する。第\ref{chap:result}章では本研究の総括を行い、また今後の展望について議論する。第\ref{chap:conclusion}章では備考を述べる。

%この第\ref{chap:introduction}章では、本論文を書くに至った背景とその構成を説明している。\\
%第\ref{chap:webapi}章では、検索エンジンの構成に使用したWebAPIと、それらを統合した手法について説明する。第\ref{chap:search}章でキーワードを用いた検索結果の精度向上手法について説明する。第\ref{chap:visualize}章では、WWW視覚化という目線での先行研究や開発事例、そして解決方法についての案を提起する。第\ref{chap:coding}章では、アプリ開発のための言語や手法についての詳細を概説する。第\ref{chap:ledoxea}章では、開発したアプリの利用方法と、その特徴について説明する。第\ref{chap:result}章では、そのアプリについての評価を行い、その結果から考察を述べる。第\ref{chap:conclusion}章では、本研究のまとめを行い、今後の課題を列挙する。	% 序論
\chapter{関連研究}
\label{chap:webapi}

本章ではユーザーの仮想現実や明晰夢における認識度や要求について事前調査、睡眠に関する調査、睡眠中に見たい夢の分析を行った。そしてDreamDateの開発に反映した点について述べる。

\section{仮想現実システムに関するアンケート調査}
人がどのような仮想現実を望んでいるかを調査するため、理系の学生7人、文系の学生7人、サラリーマン10人、主婦3人を含めた20〜60歳の男女27人にオンラインアンケートをした。これらのインタビュー結果を経て一般的なユーザのニーズを把握し、DreamDateの有効性やDreamDateが解決すべき問題について明らかにする。

%\subsection{仮想現実を体験するために一般的な人々が支払う金額}
%各所公式ウェブサイトを掲載し、機能性やデザインの詳細を説明した上で、仮想現実を見る手段として次の選択肢から購入しようと思う商品を選んでもらった。

%\begin{itemize}
%\item OCULOUS Rift:85278円
%\item ハコスコ:1500円
%\item iWink:36478円
%\item タカラトミー夢見工房:15984円
%\item DreamDate :無料
%\end{itemize}

%すると図\ref{userNeedCost}のように、一般ユーザーの中には仮想現実を体験するためにOCULOUS Riftなどの高額なデバイスを購入しようとする人は少ないということが分かった。1500円ハコスコだと少し数が増えるが、これらのデータから無料で簡単に手に入れることができるツールを多くの人が必要としていることが示唆された。

%\begin{figure}[htbp]
%\begin{center}
%\includegraphics[width=15cm]{eps/VRselection.eps}
%\caption{仮想現実を体験するために一般的ユーザーが選ぶデバイス}
% \label{userNeedCost}
%\end{center}
%\end{figure}

\subsection{仮想現実を体験したいタイミング }
仮想現実を体験したいタイミングとして、睡眠中と起きている時間帯でどちらが好ましいかについて調査を行った。すると睡眠中と答えたのは52\%、起床中と答えたのは48\%。このように結果にはあまり差がなかった。睡眠中を選択した人は理由として「睡眠時間の有効活用のため」と答えた。比べて起きている時と選択した人は「起きたら忘れてしまうかもしれないから、意識のある時に体験したい」と答えた。よってDreamDateの開発において、睡眠中の体験がユーザーにどのような印象を与えるかを研究することは意義があると考えられる。

\section{夢と睡眠}
睡眠中に扱うアプリケーションの開発にあたって、睡眠自体を理解することは不可欠である。事前調査によって明らかになった睡眠段階や夢についてここで記す。

\subsection{夢と記憶}
夢は空間、時間軸と、登場人物が非現実的な場合など不合理で異様な内容のことが多いが、大抵の場合人は、現実だと錯覚し夢を見ていることに気がつかない。それは論理的思考力を担う前頭前皮質の機能が低下しているためだ\cite{cortex}。起床後も夢での感情が現実で起きたかのように勘違いしてしまうほどリアルな体験をする人も多い。起床後すぐに夢日記をとれば、本当の思い出のように夢の記憶が残る場合もある。
 心理学者であるSigmund Freudは1905年に無意識の欲求や感情、抑圧された子供の頃の記憶、生理的欲求などが夢に大きな影響を与えていると述べた\cite{freud}。一方で2006年にJie Zhangは夢は短期的な記憶を長期的な記憶に変換するためのプロセスであると述べている。この図\ref{brainZhang}は睡眠中の脳の働きを表す\cite{Zhang}。脳の容量には限度があるため、睡眠中に過去の記憶の中で関連性の強い記憶を繋げたり、重複している内容や必要のない記憶を消しているのだ\cite{Zhang}。睡眠時間が減ると暗記能力が減るのもこれにより説明できる。\\
 幼児の平均睡眠時間は16時間でそのう内の50\%をREM睡眠が占める。一方成人の平均睡眠時間は7時間でREM睡眠も短いため、夢をあまり見なくなる。年齢が若いほどREM睡眠の周期が長いのは、経験すること全てが新しいため多くのことを記憶しなければならないことと、脳の空き容量が多いためと説明されている。

\begin{figure}[htbp]
\begin{center}
\includegraphics[width=15cm]{eps/sleepBrainModel.eps}
\caption{nonREM睡眠とREM睡眠}
\label{brainZhang}
\end{center}
\end{figure}

\subsection{睡眠と睡眠段階(睡眠の深さのレベル)}
 睡眠は身体を休めるためにある。そして人生の1/3を占める活動である。そして睡眠中人は2つの睡眠段階、レム睡眠とノンレム睡眠を90分間隔で行き来している\cite{Dement}。筑波大学と理化学研究所の研究によるとREM睡眠中は記憶形成や脳機能回復の作用がある脳波(デルタ波)が多く見られるというのが通説である\cite{tsukuba}。そしてレム睡眠中は心拍数や眼球の運動が活発化する。レム睡眠の最中に起きたときは夢も比較的覚えているという研究もされている\cite{remNonRem}。一方ノンレム睡眠中は脳も身体も休んでいる。\\
 図\ref{SleepHypnogram}は平均的な睡眠のサイクルを示したものである\cite{hypnogram}。睡眠に突入して初めてのレム睡眠は10〜12分でもっとも短い。2度目のレム睡眠は15〜20分。最後の夢は15分であるが、通常はアラームなどによって不意に中断されることが多い。平均的に一晩で5〜7回夢をみる。一般的な人生で人は6年間夢を見る。DreamDateはこの6年間をより充実感のある体験にするために貢献できるアプリケーションになる可能性があるのだ。

\begin{figure}[htbp]
\begin{center}
\includegraphics[width=15cm]{eps/SleepHypnogram.eps}
\caption{REM睡眠中の脳の働きのモデル}
\label{SleepHypnogram}
\end{center}
\end{figure}

\subsection{睡眠段階のセンシング方法}
睡眠段階のセンシング方法として正確性が高いのは脳波センサーである。しかしそれ以外のセンシング方法もある。人はREM睡眠中に眼球が活発化し心拍数が多少上がるので、眼球の運動と心拍によりセンシングが可能である。人は睡眠段階を移行させるために寝返りを行う習性があるとされている。要するに睡眠サイクルのスイッチのような働きをするのだ。\cite{negaeri}よって寝返りをモニタリングすれば睡眠段階をある程度センシングすることが可能であるということが通説である。

\section{明晰夢に関するアンケート調査}
夢の操作に成功したとしてもその夢を覚えていなければ意味がない。そこで実験を始める前に一般的に人は夢の内容を起床後どのくらい覚えているのかをアンケート調査した。夢をよく覚えていると答えた人は内容によっては覚えていると答えたのは10人、覚えていないとこ答えたのは13人、よく覚えているのと答えたのは4人であった。
%\begin{figure}[htbp]
%\begin{center}
%\includegraphics[width=15cm]{eps/remember.eps}
%\caption{夢を覚えている比率}
%\label{rememberDream}
%\end{center}
%\end{figure}

覚えている夢は刺激的、怖い夢、繰り返し見た夢というのが多く、日常的な夢は忘れがちであるということが分かった。人は睡眠中の夢の90\%を起床後5分間で忘れるという。Jie ZhangによるとREM睡眠中は短期的な記憶を担っている脳は長期的な記憶への移行に注力していて、インプットの部分があまり機能していないためであると説明する\cite{Zhang}。ただ起きてすぐに夢日記で夢を記憶すれば覚えていられることが可能なのだ\cite{forgetDreams}。\\

Sigmund Freudは「夢判断」の中で人は睡眠中の姿勢、環境、身体的刺激によって夢の内容が変化すると述べた\cite{freud}。睡眠中の人間の鼻先を羽毛でくすぐったときに、夢の内容に変化があったことを確認する実験を紹介している。そこで音、体制、匂い、振動、光、などの刺激の中で何が夢に一番影響を与えやすいのかを男女27人にオンラインアンケートをとった。以下の図\ref{externalShigeki}がその結果を示す。音が他の刺激よりも影響を与えやすいということが分かった。また学術的にも聴覚と嗅覚は人間の生命維持を高めるために感度は低いが睡眠中も機能しているということが証明されている\cite{Zhang}。\\

\begin{figure}[htbp]
\begin{center}
\includegraphics[width=15cm]{eps/input.eps}
\caption{夢に影響を与えた外的刺激}
\label{externalShigeki}
\end{center}
\end{figure}

明晰夢を体験したいか否かで質問をしたところ77\%の人が体験したいと答えた。仮想現実で体験したい内容を調査結果から似ているものをカテゴリー別に分けて、図\ref{desiredDreamTpye}で示した。LOVEタイプ、癒しタイプ、元気欲しいタイプ、アドベンチャータイプ、ストーリータイプ、ビジネスタイプとあるがそれぞれの定義を述べる。LOVEタイプとは恋愛や性的行為などが含まれる内容。アドベンチャータイプは冒険など非日常の体験を求める内容。ストリータイプはドラマのように連続性のある夢を求める内容。癒しタイプ・元気欲しいタイプは娯楽を求める内容。原強化タイプは睡眠中になんらかの学習を求める内容だ。LOVEタイプと癒し・元気が欲しいタイプが最も多く、少数派としてビジネスタイプがあった。\\

\begin{figure}[htbp]
\begin{center}
\includegraphics[width=15cm]{eps/dreamType.eps}
\caption{明晰夢で体験したい内容のカテゴリ:分析1}
\label{desiredDreamTpye}
\end{center}
\end{figure}

回答をさらに違った方法で分析した結果が\ref{desiredDreamTpye2}である。これらの結果からユーザーによって理想の夢は日常や非日常、具体性や抽象性に隔たりがあり、一貫性が見られないことがわかった。
\begin{figure}[htbp]
\begin{center}
\includegraphics[width=13cm]{eps/whatYouWantToDream.eps}
\caption{明晰夢で体験したい内容の詳細:分析2}
\label{desiredDreamTpye2}
\end{center}
\end{figure}

\section{睡眠の観測と夢の制御}
\subsection{睡眠の観測}
DreamDateは睡眠中に仮想現実を体験するための手段として考えられた研究である。よって正確に睡眠をモニタリングする方法を探究するというのはこの論文の主旨ではない。しかし明晰夢に影響を与えるのに夢をみる時間帯であるREM睡眠を観測することは重要な鍵となる。以下にモニタリングに注目した先行研究を紹介する。

Bedditは睡眠の質を向上させるためにセンサーで情報を蓄積してアプリでユーザーに情報を共有するために作られたデバイスである。マットの上にセンサーを配置、鼓動によって起きる血流の変化と呼吸に伴う肺の動きをセンサーで観測をしている。ウェラブルデバイズではないためユーザーが使用しやすいが、デバイスが18966円と高額である。
%\item 正確性:46人を対象に実験し、心電計と比べた結果bedditの結果が99.94%の相関性があると証明されている
%\item 値段:18966円

株式会社オムロンが開発したねむり時間計は、枕元に置くだけで電波センサーが睡眠時間を測定する。測定結果がスマートフォンに転送されて、アプリで睡眠の質(寝返りの回数など)や時間を一週間単位で分析できる。ユーザーに適した睡眠効率向上のアドバイスする。電波センサーでモニタリングが行われる\cite{omron}。プロダクト自体デザインやアプリのユーザーインタフェースへのこだわりが評価され2012年に『グッドデザイン賞』を受賞している。
%\item 正確性:商業目的の製品のため詳しいデーターは明かされていない
%\item 値段:3630円

\begin{figure}[htbp]
\begin{center}
\includegraphics[width=7cm]{eps/omuron.eps}
\caption{オムロン睡眠計の外観}
\label{omuron}
\end{center}
\end{figure}

neuroonは睡眠サイクルの改善、光セラピーのためのウェラブルマスクである。脳波、眼球の動き、心拍数、血液中の酸素量、加速度、体動と体温全てのセンサーが搭載されている\cite{neuroon}。脳波を含んでいるということで正確性が高いのが特徴である。
%\item 正確性:商業目的の製品のため詳しいデーターは明かされていない
%\item 値段:36478円

Zeroは睡眠時無呼吸症候群の解決などを目的としたウェラブルデバイスである \cite{beWellApp}。脳波センサーにより睡眠の各ステージを正確に突き止めることができ、正確性は高い。しかし頭に装着しなければならないデバイスなので汗をかきやすくなり、ユーザーの負担になるので長期的な利用には向いていない。

iSleepは健康向上のために睡眠時間とその質をスマートフォンのアプリである。スマートフォンに備わっている音声録音機能で体動、咳やいびきなどを測定\cite{iSleep}。アプリをダウンロードするだけで簡単に使えるが特徴だ。ローンチ10日間で100人のユーザーから睡眠に関する詳細なデータを集めている。
%\item 値段:361円
%\item 正確性:被験者7人51日間の睡眠で90\%の正確性

BeWellAppもうつ病、心配性、不眠症、高血圧になりにくい生活習慣へ導くために睡眠の長さを測るスマートフォンのアプリである。しかしユーザーによるインプットは一切必要なく、ユーザーの充電、加速度からスマートフォン利用頻度・時間を測定、静けさ、部屋の明るさなどから、睡眠スタイルを検知する仕組みになっている\cite{beWellApp}。アプリという形で多くの人に実験をしてもらえる、ユーザーは普段の生活となんら変わりなく、過ごせるため、負担がかからないのが特徴だ。
%\item 効果:8人の被験者に、Jawbone、Zeoと、BESモデルのアプリを試してもらい、全ての人がユーザー体験を過ごせたと結果がきた。
%\item 正確性:睡眠時間+-42分
%\item 値段:商業用目的ではないため不明

以上の先行研究を踏まえると、睡眠の感知の仕方は様々であるということがわかる。最も正確であるとされているのは眼球の動きをトラッキングする手法と脳波センサーであるが、これでは身につけるタイプのセンサーなのでユーザーの負担になってしまう。またデバイスを購入するためのコストがかかってしまうので、本論文の主旨とずれてしまうことになる。次に体動検知のために使われる電波センサーであるがこれもデバイスの購入を必要とする。そこでもっともユーザーの負担ならずに正確性も証明されているのがiSleepをはじめとするスマートフォンアプリケーションだ。こられの理由からスマートフォンアプリケーションが問題解決の手段としてもっとも適していると判断した。そこでiSleepで紹介されているアルゴリズムを参考にしたプログラムを製作した。

\subsection{睡眠時の刺激提示}
 外的刺激を与えることで睡眠に影響を与えようとした研究や製品開発は前例がある。首都大学の長塚麻美らによる研究では睡眠深度に即した光の刺激を与える抱き枕型のインタフェースを提案している\cite{sleepSheep}。寝付きやすくするために赤に近い黄色の光を点灯させ、波の音を再生するというシステムであるがその実験結果は明らかにされていないためさらなる実験が求められる。\\
 株式会社タカラトミーが開発した夢見工房は音、香り、視覚的情報により夢をより理想的なものに近づけるデバイスである。具体的にはユーザーが寝る前に「恋愛」、「勇気」や、「冒険」からみたい夢のテーマを選び、睡眠時にそれに紐ずいた音楽と香りが発生する。またボイスレコーダー機能もついており、みたい夢を暗示する声が目覚めない程度の小さな音量で自動的にリピート再生させることもできる。 \cite{takaratomi}このように多くの機能が備わっており、気分転換や充実した楽しい時間を作り出すことを目的としているが、正式な効果は発表されていない。また香りや音声の種類に限りがあることや香り製造機の騒音が問題としてあげられている。値段も14,800円と高めである。
\begin{figure}[htbp]
\begin{center}
\includegraphics[width=14cm]{eps/takaratomi.eps}
\caption{夢見工房}
\label{takaratomi}
\end{center}
\end{figure}

 他にも明晰夢を促進するとためのツールとしてiWinksにより開発されたAURORAがある。\cite{iWinks}これはレム睡眠時に光による刺激を与えることで、ユーザーに夢を見始るという暗号を送って明晰夢を促すデバイスである。脳波センサー(EEGセンサー)と加速度センサーが組み込まれており、睡眠の質観測においてはクリニックにより検証されたものとウェブページ上には書かれている。しかし明晰夢への効果に関する実験結果が明らかになっていないので、真実であるという確証はない。値段も3,6000円と非常に高額である。\\
 スマートフォンアプリ部門ではDreamOnがある。睡眠中に音楽を流すことで夢に影響を与えるアプリだ。アプリという形で多くの人に夢の研究に参加してもらい音が夢に影響を与えるのか否かについて判明することを目的としている \cite{dreamOn}。ウェブサイトには睡眠は外的刺激に影響を受け、森林の音を流すと夢の中に緑が頻繁に現れて、街中の音を流すと奇抜な夢を見ると書いてある。しかし実験結果についてはかなり疑わしい。App Storeでユーザによる評価は385人によってレーティングが行われており、評価は5点中3である。下記の画像\ref{DreamOnImage}はユーザーのレビューである。悪夢を見たなどの睡眠被害を訴えるレビューが多く見受けられた。DreamOnの他にもユメミール \cite{yumemiru}やDreamDream \cite{DreamDream}など国内のアプリもあるが、どれも信ぴょう性のある実験結果は公開されていない。

\begin{figure}[htbp]
\begin{center}
\includegraphics[width=5cm]{eps/dreamOn.eps}
\caption{DreamOnのユーザレビュー}
\label{DreamOnImage}
\end{center}
\end{figure}

\section{本研究の位置付け}
Head Mounted Display(HMD)は仮想現実を体験するためにメジャーな手法として注目を浴び、開発が進んでいる。しかし体験をするためにはOCULUSを始めとした高価なデバイスを購入しなければならない。3 次元のコンテンツを作成するには技術やコストが高くユーザが望むコンテンツを気軽に作れるようにはなっていない。

そこでDreamDateは明晰夢に着目した。明晰夢は睡眠という習慣をより有効に活用し、金銭的コストをかけることなく遂行することができる。1章でも紹介したが明晰夢を体験するためのステップとしてMnemonic Induction of Lucid Dreams (The MILD Technique)がある。しかしThe MILD Technique には労力が必要で誰もが気軽に始められるものとは言い難い。HMDのように金銭的負荷はかからないが快適なユーザー体験ではない。DreamDateはMILDよりにも簡単に使い始められるように、スマートフォンアプリによってユーザーのサポートをする。夢を忘れない体質になるためにThe MILD Techniqueのステップの1つとして紹介されている夢日記をアプリの機能に組み込んだ。

また明晰夢を促進するシステムとして株式会社タラトミーによる夢見工房やDreamONなどのスマートフォンアプリなどが開発されているが、実験結果については明らかになっていないのが現状だ。これらの多くはレム睡眠を検出して音声や香りによる刺激を与えることで夢に影響を与えるシステムだ。DreamDateの開発を通してどのような刺激の与え方最も有効的なのかを調べる。詳細は第3章と第4章で述べる。またDreamDateはbeddit\cite{beddit}、Bed Prssure MatやiSleep\cite{iSleep}などのように睡眠モニタリングデバイスを参考にしてモニタリングシステムの構築をする。

DreamDateシステムに求められる要件:
\begin{itemize}
\item 明晰夢で仮想現実を体験できる
\item ユーザ一人一人の要望に合った音を選ぶシステム
\item HMDのように高価なデバイスを必要としない
\item MILD Techniqueと違って、負担の少ないユーザー体験
\end{itemize}

%これまでのシステムとの共通点と相違点を明確に述べること
% 旧2.4 調査から分かったことの内容は、「システムに求められる要件」としてこの中に書いた上で、これまでのシステムとの共通点と相違点を明確に述べること
%論文ではここが重要、添削されたアブストラクトをベースにして量を増やすこと
 
 	% WebAPIについて
\chapter{DreamDateのプロトタイピングと機能}
\label{chap:search}


この章ではではDreamDateがスマートフォンアプリで睡眠中に音を流すという形に至った背景を述べる。

\section{刺激提示のプロトタイピング}
 人には視覚、聴覚、触覚、味覚、嗅覚を含む5つの感覚器がある。本研究ではそのうちの聴覚と嗅覚による刺激が睡眠中の夢に与える影響を実験を通して観察した。

\subsection{香りによる刺激}
\begin{figure}[htbp]
\begin{center}
\includegraphics[width=9cm]{eps/smell.eps}
\caption{香りによる刺激}
\label{smell}
\end{center}
\end{figure}

 ラベンダーやバラのような良い香りは睡眠に良い影響を与え、心地良い夢を見やすくするということはMichael SchredlとBoris Stuckの研究によって証明されている\cite{roseDream}。しかし香りが夢の内容に影響を与えるか否かの研究はまだ行われていない。そこでこのプロトタイプはREM睡眠の時に思い出と直結する香りを出して夢を刺激することで夢になんらかの影響を与えられるものか否かを確かめるために製作した。\\
 加速度センサーでREM睡眠を検出したらアロマランプに光がつき、5分後香りが部屋中に充満するという作りになっている。図\ref{smell}にあるのはそのプロトタイプの写真だ。\\
 実験に参加したのは嗅覚が正常に機能している(風邪などを引いていない)22歳の女性3名だ。被験者1には交際相手が部屋で使っているアロマとコーヒー豆の香りで刺激した。被験者2と被験者3はコーヒー豆の香りで刺激した。その香りをたくとすぐに過去の思い出と直感的に繋がる香りをあえて選んだ。またアロマライトは被験者の頭のすぐ横に置いた。\\
 2015年の1月に10日間の実験を行った。香りありの夜、香りなしの夜を5日間ずつ交互に繰り返した。その結果が以下の図\ref{smellExperiment}の通りだ。

\subsection{音による刺激}
 同じ被験者に今度は香りではなく音によるインプットをしてもらった。REM睡眠中に海の音や交際相手と一緒に聞いた音楽を流した。すると音によっては起こされてしまったり、被験者によっては全く影響が出ないという結果になった。しかし、図\ref{smellExperiment}が示すように、香りのインプットでは影響が全くなかったのに比べ、音のインプットは被験者1、被験者2ともに音による刺激で1日だけ夢を観たことがわかった。

\begin{figure}[htbp]
\begin{center}
\includegraphics[width=15cm]{eps/smellExperiment.eps}
\caption{香りによる刺激の実験結果}
\label{smellExperiment}
\end{center}
\end{figure}

\section{睡眠観測のプロトタイピング}
ユーザーの睡眠深度のモニタリング方法はいくつかある。それぞれの方法をユーザビリティと機能性の2つの観点から、実験を通して分析する。

\subsection{脳波センサーによる観測}
 このプロトタイプではNeuroSkyのThinkGear ASICモジュールという脳波センサーを使用した。Theta波が4〜7.2HzかつDelta波が0.5〜4Hzである時をREM睡眠中であるとし自らが実験台となり装着して寝てみた。しかしこの手法は頭を締め付けられる感覚があり、且つ汗をかいてしまうのでユーザーに負担がかかる。寝心地を損ねてしまうということがわかりセンサーの体と離す別の方法を試すことにした。
\begin{figure}[htbp]
\begin{center}
\includegraphics[width=10cm]{eps/brainWave.eps}
\caption{脳波センサーによるセンシングのプログラムと睡眠ステージと脳波の数値}
\label{brainWave}
\end{center}
\end{figure}

\subsection{心拍センサーによる観測}
 次に市販で売られている心拍センサーを追懐睡眠中の心拍数を観測することでREM睡眠を検出できるかどうかの実験をした。しかし寝ているときに指にセンサーを装着するのは発汗のを引き起こし、ユーザー体験の視点から非常に好ましくないということがわかりまたしても別の方法を試すことにした。

\begin{figure}[htbp]
\begin{center}
\includegraphics[width=10cm]{eps/heart.eps}
\caption{心拍センサーによるセンシング}
\label{heart}
\end{center}
\end{figure}

\subsection{kinectによる観測}
 このプロトタイプはKinectを使用して、ユーザーの寝返りを検知して音楽を流すシステムである。図\ref{kinect}のようにkinectを天井に設置する。ウェラブルセンサーではないためユーザには負担がかからない。但し布団をかぶってしまうとkinectによる骨格トラッキングは難しい。そのためOpenCVのライブラリを利用して、画像処理を行った。\\
 寝返り判定の正確性を確かめるために、実際にベッドの上で寝返りを打ったとき音が鳴るかを試したところ、開発したプログラミングではノイズが多く出て誤作動が起きてしまうのでkinectを使うのは適切ではないと判断した。しかしプログラミングの能力が高い人により開発されれば、kinectによるトラッキングの精度もあげられるはずである。ただし、デバイス自体の価格が高いのと、取り付けに労力が必要とされることと、ポータブルではないため旅先では使えないという点で、本研究では好ましくないとした。

\begin{figure}[htbp]
\begin{center}
\includegraphics[width=15cm]{eps/kinect.eps}
\caption{kinectによるセンシング}
\label{kinect}
\end{center}
\end{figure}

\subsection{スマートフォンの加速度センサによる観測}
 最終的に多くの人々が既に使用していて、ユーザビリティーの視点から見てもっとも負担のかからないスマートフォンアプリケーションによるセンシングに試みた。スマートフォンで計測すときはウェラブルではないため身軽であるし、持ち運びが簡単なので旅中も使える。スマートフォンアプリケーションによるセンシング方法とその正確性については5章で述べる。

\section{実装}
\subsection{機能}
 現在、仮想現実を体験すべくヘットマウントディスプレーなどの様々なツールが開発されている。そこで睡眠中の夢を自由自在にコントロールする方法があれば誰もがより簡単に仮想現実を体験できるのではないかと考えた。\\
 睡眠中にユーザーがの思い出と関連した音を流すことで、その音に基づいて夢を見ることを促進するスマートフォンアプリDreamDateを試作した。ターゲットと考えているユーザーは日々のストレスから解放されたい人、懐かしい思い出をもう一度体験したい人、物理的に会えない人と会いたい人などだ。\\
 DreamDateには3つ主要な機能がある。一つ目は寝る前に印象に残っている記憶に関する写真と映像を表示する機能。二つ目は睡眠中にREM睡眠を検出し、記憶を連想させる音を流す機能。例えば特別な誰かを連想する音、旅行中によく聞いていた曲、最寄り駅の音楽、好きな映画のサウンドトラックなどだ。三つ目は起床後に夢について記録する夢日記機能である。ユーザーには睡眠前にスマートフォンを画像\ref{DreamDateImage}のように枕の横に置いてもらう。\\
 アプリを使用して実験をした結果、DreamDateには欠点があることがわかった。それはユーザーが自分の記憶を連想する音を探し出して登録しなければならないということ。例えば海の音を聞けば海の夢を見れるということではないのだ。しかし実際に海に行った特別な記憶がある人であれば、海の音を流せばその夢を見る確率は比較的に上がる。\\
  DreamDateは開発途中でまだAppストアには掲示していないが、githubからソースコードを入手することができる。iOSスマートフォンを持っていて、Apple Developerの登録をしている人であればインストールできるようになっている。

\begin{figure}[htbp]
\begin{center}
\includegraphics[width=14cm]{eps/dreamDate02.eps}
\caption{DreamDateの配置}
\label{DreamDateImage}
\end{center}
\end{figure}


 Xcode 上で openframeworks ライブラリを利用して、iPhoneの加速度センサーを利用した体動検知アプリケーションを制作した。睡眠時に枕の横に iPhone を置いて、体動(寝返り)による寝具の動きを検知して加速度を測定す る。\\

 まずベッドの硬さは人により違うため、キャリブレーションをしてもらう。アプリ起動後ユーザーにはiPhoneを横に置いた状態で15秒間静止してもらう。x軸の加速度を毎秒記録、1秒前の加速度との差分を導き出す。20秒間、x軸の差分の中での最大値を閾値として設定する。y軸とz軸の測定をしなかったのはx軸だけでも十分寝返りを特定できるためである。\\

 ベッドで寝てから睡眠に至るまで平均的に10分から20分かかるとされているため、スタートボタンが押されてから20分後に加速度センサーによる体動のモニタリングが開始される。こうすることで、寝ようとしている最中に音楽がならないようにする。モニタリングが開始されてからはノイズを除去するために、毎20秒の平均値が出される。その平均値が閾値に比べて高くなった時に寝返りをしたと判定する。寝返りを打つ時は睡眠段階がREM睡眠からnonREM睡眠に、あるいはnonREM睡眠からREM睡眠への切り替わったときだ\cite{negaeri}。そのためREM睡眠の間はあらかじめ設定していた音楽が図\ref{melodyGraph}で示したオレンジのタイミングで流れる。REM睡眠時にのみ音楽を流したのは常に音楽が流れていると睡眠が害され睡眠サイクルが崩れて体調不良などを引き起こす可能性が高くなるためだ。

\begin{figure}[htbp]
\begin{center}
\includegraphics[width=15cm]{eps/remNonrem.eps}
\caption{音刺激提示のタイミング}
\label{melodyGraph}
\end{center}
\end{figure}


一晩中のx軸の数値、音楽の再生状況、夢日記の結果はデータベースをクラウドであるParseに保存する。図\ref{system}に一連のプログラムの流れを記載する。
\begin{figure}[htbp]
\begin{center}
\includegraphics[width=15cm]{eps/system.eps}
\caption{DreamDateのフローチャート}
\label{system}
\end{center}
\end{figure}

\subsection{利用方法}
 ユーザーには予め記憶を思い起こさせる音と画像を登録してもらう。音選びは適している音声と適さない音声があるため注意する必要がある。使用してはいけない音は人の声だ。特に喋りかけてくるような内容の音声は、ユーザーを起こしてしまう可能性が高いということが実験結果から分かった。詳しくは第6章で述べる。逆に適している音は繰り返しある環境下で聞いていた音である。
 アプリの起動後、\ref{le01}のような画面が表示される。そこには「自動ロック機能をOFFにする」「音量は1〜3に設定する」やiPhoneの置く位置などの指示が書かれている。次に\ref{le02}の画面に遷移し、ユーザーの思い出に関連性のある画像を表示する。ここでは寝る前に記憶の情景を思い出す機会を与えている。そして\ref{le03}の画面では思い出の音楽が流れる。音楽を聴きながら、旅先での空間、香り、音の細部までを思い出して、気持ちを落ちつかせて瞑想状態に入ってもらう。次に\ref{le04}の画面に移動する。ユーザーは寝る前にアプリを起動してスタートボタンを押し、起動させたままスクリーンを伏せて枕の横に置く。20〜30分間後にDreamDateの加速度が起動をし一晩中ユーザーの体動のトレッキングが行われ、REM睡眠を検知すると音楽がなる。起床後\ref{le05}の画面で、ユーザーは起床すると夢の内容を忘れないように日記に投稿する。

\begin{figure}[htbp]
 \begin{minipage}{0.45\hsize}
  \begin{center}
   \includegraphics[height=90mm]{eps/AppIntro.eps}
  \end{center}
  \caption{起動画面}
  \label{le01}
 \end{minipage}
 \begin{minipage}{0.45\hsize}
  \begin{center}
   \includegraphics[height=90mm]{eps/AppMemoryImages.eps}
  \end{center}
  \caption{思い出の画像を表示}
  \label{le02}
 \end{minipage}
\end{figure}

\begin{figure}[htbp]
 \begin{minipage}{0.45\hsize}
  \begin{center}
   \includegraphics[height=90mm]{eps/AppMusicPlay.eps}
  \end{center}
  \caption{思い出に関連した音刺激の提示}
  \label{le03}
 \end{minipage}
 \begin{minipage}{0.45\hsize}
  \begin{center}
   \includegraphics[height=90mm]{eps/AppStart.eps}
  \end{center}
  \caption{睡眠開始ボタン}
  \label{le04}
 \end{minipage}
\end{figure}

\begin{figure}[htbp]
 \begin{minipage}{0.45\hsize}
  \begin{center}
   \includegraphics[height=90mm]{eps/AppDiary.eps}
  \end{center}
  \caption{夢日記記入ページ}
  \label{le05}
 \end{minipage}
 \begin{minipage}{0.45\hsize}
 \end{minipage}
\end{figure}
	% 検索エンジンの精度向上
\chapter{ユーザスタディ}
\label{chap:visualize}

本章では開発したスマートフォンアプリDreamDateを用いたユーザスタディとその結果について述べ、提案手法の長所及び短所について考察する。

\section{予備実験1:音刺激の有無}
寝る前の10分間とレム睡眠中に曲を流すことが夢に影響を与えるか否かを検証するため予備実験を行った。スマートフォンは充電をした状態で枕の横に置くことで脳が音に反応しやすい状態にした。また睡眠を始める前に5分間海の夢が見たいと被験者に念じてもらった。\\
 20代後半女性の被験者A、40代後半女性の被験者Bと、20代前半女性の被験者Cに海の波の音を聞く日と聞かない日を交互に14日間続けてもらうことで音が夢に影響を与えるのか否かの記録を行った。MILDのトレーニングを発案しその効果を実証したDenysでも明晰夢を週4回明晰夢を体験できたら多い方であるという\cite{LaBerge}。なぜならば明晰夢はその日のスケジュール、体調や、心境の影響を受けて睡眠が不規則った時などは夢を忘れたり、悪夢を見たりするためである。DreamDateはMILDと比較できるように14日間という期間の実験を行い、どのくらいの成功率でどういった内容の夢をみることができるのか検証した。図\ref{experiment1}が実験スケジュールと実験結果である。青のハイライトがある日が関連する夢を見た日である。音が無い場合に被験者が海の夢を見たのは1回なのに対し、海の音を流して海の夢を見たのは4回であった。\\
 夢の具体的な内容について実験後インタビューを行った。すると3日間夢を見たと答えた被験者Aは音のインプットが無い日は会社で働いている夢を見ることが多く、音を流しながら寝た日は10日ほど前に行った沖縄旅行での夢を見たと答えた。一度も海に関連した夢を見なかった被験者Bは海の音で起こされたりしたため、音は流れていたが全く関係の無い夢を見たと答えた。被験者Cは実験の最後の方で1年前に旅行したアメリカ西海岸に関する夢を見たと答えた。

\begin{figure}[htbp]
\begin{center}
\includegraphics[width=13cm]{eps/schedule0.eps}
\caption{予備実験1:実験スケジュールと実験結果}
\label{experiment1}
\end{center}
\end{figure}

\section{予備実験2:音刺激の種類}
どのような音がDreamDateに適しているのかを調べるために予備実験を行った。2年前から遠距離恋愛中の交際相手とデートをしている夢を見たいと望む被験者Cに「音声」、「曲」と「自然音」の3種類の音を試した。被験者Cの場合は「音声」は交際相手が被験者Cの名前を語りかけ、過去のデートの思い出話や、理想のデートの話しや、愛の言葉をささやくといった内容であった。「曲」は交際相手が被験者Cのために作曲と演奏した曲で、被験者Cも毎日通学で聞いている曲である。「自然音」はアメリカ西海岸の海で交際相手と共に聞いた波の音。図\ref{experiment2}が実験スケジュールと実験結果である。被験者の希望であった遠距離恋愛中の交際相手とデートしたいという要望を実現させることができたのは「曲」であった。\\
 夢の具体的な内容について実験後インタビューをした。関連する夢を見たのは「曲」と「自然音」のときである。11月9日は交際相手と日本で再開する夢をみて、11月13日は江ノ島で友人と遊ぶ夢をみた。11月15日は交際相手から手紙とプレゼントが届く夢を見た。語りかけ口調の音声は被験者を毎回起こしてしまった。比べて波の音などの自然音や音楽は比較的被験者を1度しか起こさなかった。

\begin{figure}[htbp]
\begin{center}
\includegraphics[width=10cm]{eps/schedule1.eps}
\caption{予備実験2:実験スケジュールと実験結果}
\label{experiment2}
\end{center}
\end{figure}

\section{予備実験から浮かび上がった仮説}
予備実験1を経て同じ海の波の音でも結果に個人差が出た。実験を行った後にそれぞれの被験者にどのような夢を見たかインタビューをした。すると海の夢を見た被験者 A と被験者 C はどちらも過去に海に行った際の思い出に関する夢をみたと答えた。一方で海にしばらく行っていない被験者 B は一度も海の夢をみなかった。この結果は被験者の思い出と関連性の高い音を流すと効果が出やすいということが示唆する。またその思い出がより被験者にとって印象深いもので、新しい思い出の方が夢で再現しやすい。言い換えると実世界で体験したことのないことを DreamDate で体験することは困難である。\\
 次に要素として考えられるのは年齢である。Zhangは高齢になるにつれてレム睡眠の間隔が短くなると述べているため、年齢が高いほど明晰夢を体験できる確率が下がる可能性がある\cite{Zhang}。次に考えられるのが被験者がその音に関連した夢を見たいと望む気持ちの強さである。海の音を流した夜、被験者には全員海の夢を見たいと念じながら寝るようにと伝えたが、本心として海の夢を見たいと思っていたのは被験者Aと被験者Cのみであった。夢に対する希望が大きいほど、夢を見る確率が上がる可能性がある。他にも要素として考えられるのが、音と被験者の記憶との関連度だ。最後に考えられるのが、DreamDateの使用継続日数だ。というのは被験者Aと被験者Cは共に実験の前半に比べて実験後半の方が夢をみる確率が上がっているからである。明晰夢習得へのステップであるThe MILD Techniqueでも訓練を長く続ければ続けるほど、成功率があがると紹介されていることからDreamDateもな学使えば使うほど効果が出やすい可能性が高い\cite{LaBerge}。

予備実験からDreamDateの効果に関連している可能性がある要素を以下に提示する。\\
夢制御システムDreamDateで流す音に関して:
\begin{itemize}
\item 要素1 : 思い出に関連した曲か否か(その思い出自体が印象深いものであるのか否か・いつの思い出か)
\end{itemize}
ユーザ側の精神的や身体的要素として:
\begin{itemize}
\item 要素2 : ユーザの年齢
\item 要素3 : ユーザがその夢を見たいと思う気持ちの強さ
\item 要素4 : ユーザのDreamDateの継続使用日数
\end{itemize}

\section{実験: 音とタイミング}
 予備実験から浮かび上がったDreamDateの効果に関連している可能性がある要素について検証するために再度7名の被験者を対象に実験を行った。まず仮説1を検証するために被験者は20代、30代、40代と50代の人々を集めた。仮説2に基づいてすべての被験者に本当に見たい夢の内容を前もってインタビューで聞き、その思い出に由来する音を聞き出しDreamDateに反映した。仮説3を明らかにするために頃の記憶か、直接的な記憶なのか、間接的な記憶なのかを事前にインタビューで聞き出し、検証するための材料とした。仮説4を明らかにするために15日間実験を行い、時間の経過と実験結果に相関性があるかを検証した。\\
 被験者には6時間以上睡眠を取れる日にのみ実験に参加してもらった。被験者にはThe MILD Techniqueに基づいて夢を記憶できる体質になってもらうために実験を開始する5〜10日間前から、被験者には夢日記を書いてもらった。加えて寝る前に音楽を聴きながら思い出に関する画像を5分間眺めること、思い出について考えならが寝る意識をしてもらった。予備実験1と2では被験者が音に起こされてしまうという事態が発生したので本実験では、レム睡眠中ずっとと起きる直前のレム睡眠のみに音を流す場合をそれぞれ検証した。具体的には1 日目は音楽なし、2 日目はレム睡眠中音楽あり、3 日目は起きる直前のレム睡眠中音楽ありというを繰り返し 5 回、合計 15 日間続けてもらった。
 図\ref{allRemLastRem}は被験者6の12月20日(レム睡眠中ずっと音を鳴らした日)と12月21日(起きる前のレム睡眠のときだけ音を鳴らした日)の加速度のデータと音が鳴っていた時間帯の記録を示すデータである。

\begin{figure}[htbp]
\begin{center}
\includegraphics[width=15cm]{eps/allRemLastRem.eps}
\caption{被験者6の12月20日と12月21日の加速度の変化率と音が鳴っていた時間帯を示すグラフ}
\label{allRemLastRem}
\end{center}
\end{figure}

\subsection{被験者の詳細と使用した音}
DreamDateは日本人のみならず、全世界のユーザを対象として制作しているため国籍と性別共に多様性のある被験者を集めた。また比較的安定したの睡眠活動をしている人を対象にした。下記に具体的な被験者の情報と使用する音を提示する。\\

被験者1:
\begin{itemize}
\item 国籍:インドネシア人
\item 性別:男性
\item 年齢:30代後半
\item 明晰夢の経験:5回ほど
\item 夢日記を付けた日数:5日間
\item 思い出に由来する音楽:被験者1は音楽にあまり関心がなく、特に思い出に残る音・音楽がなかった。そこで日常生活の中で音の刷り込みをした。具体的には実験10日間前から毎日コーヒーを飲むときにEdith Piafによる"Non je ne regrette rien"という曲。この音楽は映画inceptionの中で夢から覚めるために主人公たちが聴く音楽としても知られている。
\end{itemize}

被験者2:
\begin{itemize}
\item 国籍:日本人
\item 性別:女性
\item 年齢:40代後半
\item 明晰夢の経験:なし
\item 夢日記を付けた日数:10日間
\item 思い出に由来する音楽:被験者2は30年ほど前の結婚式で流した音楽 The CarpentersによるWe've only just begun
\end{itemize}

被験者3:
\begin{itemize}
\item 国籍:日本人
\item 性別:男性
\item 年齢:50代前半
\item 明晰夢の経験:なし
\item 夢日記を付けた日数:10日間
\item 思い出に由来する音楽:007の映画の中で使われているサウンドトラック
\end{itemize}

被験者4:
\begin{itemize}
\item 国籍:アメリカ人
\item 性別:男性
\item 年齢:20代前半
\item 明晰夢の経験:5回以上
\item 夢日記を付けた日数:5日間
\item 思い出に由来する音楽:宮崎駿の映画である「魔女の宅急便キキ」の中で使われているサウンドトラック
\end{itemize}

被験者5:
\begin{itemize}
\item 国籍:日本人
\item 性別:女性
\item 年齢:20代後半
\item 明晰夢の経験:なし
\item 夢日記を付けた日数:5日間
\item 思い出に由来する音楽:今年の9月に社会人ダンス部でダンスを披露したときに利用したCell Block Tangoという曲
\end{itemize}

被験者6:
\begin{itemize}
\item 国籍:アメリカ人
\item 性別:男性
\item 年齢:20代前半
\item 明晰夢の経験:5回ほど
\item 夢日記を付けた日数:5日間
\item 思い出に由来する音楽:高校時代に演奏したバンドの曲、Fountains of WayneによるStacy's Momという曲
\end{itemize}

被験者7:
\begin{itemize}
\item 国籍:アメリカ人
\item 性別:男性
\item 年齢:20代後半
\item 明晰夢の経験:5回以上
\item 夢日記を付けた日数:5日間
\item 思い出に由来する音楽:1年前に交際相手のために作曲・演奏した曲Love From The Other Side Of The World\end{itemize}

\subsection{実験結果}
図\ref{experiment3}は実験のスケジュールと実験結果である。結論から述べると被験者7人が全員1回以上関連する夢を見ることができた。また予備実験1と本実験の結果から10人中8人が、音で刺激を与えた夜の方が与えなかった夜に対して明晰夢を見る確率が高かった。図\ref{musciOnNo}がその実験結果をまとめたものである。

\begin{figure}[htbp]
\begin{center}
\includegraphics[width=13cm]{eps/schedule2.eps}
\caption{実験スケジュールと実験結果}
\label{experiment3}
\end{center}
\end{figure}

\begin{figure}[htbp]
\begin{center}
\includegraphics[width=6cm]{eps/musicOnNO.eps}
\caption{音有無の結果}
\label{musciOnNo}
\end{center}
\end{figure}

そこで予備実験を経て浮かび上がった、明晰夢に影響を与える可能性のある要素ごとに実験結果を図\ref{categorizedData}にまとめた。被験者1、被験者2、被験者5、被験者6、被験者7は直接的な体験であるのに対して、被験者3と被験者4は映画を通した間接的な思い出なので間接的と分類した。また思い出から経過している期間も記載した。

\begin{figure}[htbp]
\begin{center}
\includegraphics[width=13cm]{eps/categorizedData.eps}
\caption{実験結果のまとめ}
\label{categorizedData}
\end{center}
\end{figure}

図\ref{age}は被験者の年齢と関連する夢を見た回数の相関性を表す図である。直接的な思い出に関連した曲を聴いて寝た被験者は平均的に3.6回夢を見た。比べて間接的な思い出に関連した曲を聴いて寝た被験者は平均的に3.5回夢を見た。加えて予備実験前半に比べて後半の方が夢を見た回数が多いということが図\ref{experiment3}から読み取れる。また音の影響によって7人の被験者のうち3人が起きてしまう事態が発生することが分かった。図\ref{result}は音楽を流したタイミング別に結果を表したグラフである。全ての被験者の実験結果を合計しタイミング別に関連した夢を見たときの確率を導き出した。音を流さないときは1/35、レム睡眠中ずっと音を流したときは10/35、そして起きる直前のレム睡眠中に夢を見たときは13/35の確率で関連する夢を見た。

\begin{figure}[htbp]
\begin{center}
\includegraphics[width=12cm]{eps/age.eps}
\caption{被験者の年齢と関連する夢を見た回数}
\label{age}
\end{center}
\end{figure}

\begin{figure}[htbp]
\begin{center}
\includegraphics[width=6cm]{eps/result.eps}
\caption{刺激提示のタイミング別の関連した夢を見た回数}
\label{result}
\end{center}
\end{figure}

\section{考察}  
本論文では研究の目的を
\begin{itemize}
\item 睡眠中のユーザに音刺激を提示することで夢に登場する人物、過ごす空間、活動内容を制御 できるか
\item DreamDateによってHMDよりでユーザ個別の経験に関連する現実世界に近いコンテンツを体験できるか
\end{itemize}
を明らかにすることと設定したが予備実験と実験結果からこれらについて関連する事象について考察を行う。

\subsection{睡眠中のユーザに音刺激を提示することで見る夢の内容を制御できるのか} 
被験者7人が全員1回以上音刺激に関連する夢を見ることができた。また予備実験1と本実験の結果から10人中8人が、音で刺激を与えた夜の方が与えなかった夜に対して明晰夢を見る確率が高かったことが図\ref{musciOnNo}から分かる。この結果はDreamDateの夢制御システムとしての有効性を示す。ただ結果には明らかに個人差があった。その個人差を引き起こしている原因として考えられるのが、先にも述べた以下の要素である。

DreamDateの機能として:
\begin{itemize}
\item 要素1 : 思い出に関連した曲か否か(その思い出自体が印象深いものであるのか否か・いつの思い出か)
\end{itemize}
ユーザ側の精神的や身体的要素として:
\begin{itemize}
\item 要素2 : ユーザの年齢
\item 要素3 : ユーザがその夢を見たいと思う気持ちの強さ
\item 要素4 : ユーザのDreamDateの継続使用日数
\end{itemize}

それぞれの要素がDreamDateの効果に個人差をもたらす要因となっているのか否かを考察する。\\
「思い出に関連した曲か否か」がDreamDateの効果に影響を与えているか否かに関してはある程度検証できた。実験では合計10回REM睡眠中に音で刺激を与えた。すると自ら作曲した歌、演奏した曲、ダンスをした曲などの直接的な記憶にを連想させる音楽を流した被験者5(3ヶ月前にダンスを披露したときの音楽を流した)と被験者6(3年前にバンドで演奏したときの音楽を流した)と被験者7(1年前に作曲した音楽を流した)は平均して5回関連性のある夢を見た。次に過去の思い出に関連した音楽を流した被験者1(毎日コーヒーを飲む際の音楽を流した)と被験者2(30年前の結婚式のときの音楽を流した)は平均して1.5回関連性のある夢を見た。一方実際に自分は体験していないが映画などを通して間接的に体験した記憶にまつわる音楽を流した被験者3(1ヶ月前に見た映画の音楽を流した)と被験者4(1週間前に見た映画の音楽を流した)は平均して3.5回夢を見た図\ref{experiment3}。この結果は流す音がユーザにとってより印象深い体験ほど夢に影響を与えやすいということを示唆している。Zhangは夢は記憶の整理をするために見るものだと述べている\cite{Zhang}。被験者Aの場合波の音が過去の記憶を思い出させる音を流すことがトリガーとなり、脳が反応し思い出が夢として再生されたと考えられる。第2章で紹介したDreamOnやユメミールなどのスマートフォンアプリケーションは「鳥の音で森林にいる夢」、「貨幣が落ちる音でお金持ちになる夢」、「拍手の音で表彰される夢」、「フライパンでベーコンが焼ける音で朝食」の夢をみることができると説明している。しかしユーザはそれぞれ違った経験の持ち主なので全てのユーザにその効果が現れるかの見解には再考の余地が残る。株式会社タカラトミーのホームページでも夢見工房の効果に関しては個人差があると紹介している。その理由もユーザによって音や香りとそれまでの記憶との関連性が違うからだと推測することができる。\\
 実験後「年齢」がDreamDateの効果に影響を与えているか否かに関しては、30歳以下の被験者は平均的に4.75回明晰夢を体験したのに対して、30歳以上の被験者は平均的に2回明晰夢を体験したという結果から、年齢が上がるに連れて明晰夢を体験しにくい可能性がある。しかし被験者の数が少なすぎて図\ref{age}のグラフからは年齢と結果に十分な相関を見ることはできなかったため、被験者を増やしてもう一度実験をする必要がある。\\
 実験では「その夢を見たいと思う気持ちの強さ」がDreamDateの効果に影響を与えているか否かに関しては検証できなかった。それはそれぞれの被験者がどのくらいその夢を見たいかを数値的に計ることが難しかったためである。再度実験を行う場合は事前に明晰夢に対する期待度を数値化する必要があるだろう。例えば「夢をすごく見たい」「夢を見れたら嬉しい」「夢を見れなくてもまぁ仕方ない」のどれかを選んでもらい、被験者が音に関連する夢を見たかの回数と被験者の期待度に相関があるか否かを検証する必要がある。\\ 
「 DreamDate の継続使用日数」がDreamDateの効果に影響を与えているか否かに関しては検証できなかった。図\ref{dates}は図\ref{experiment3}の結果からDreamDateを使用しなかった日数を除いてグラフにした、合計10日間の実験と明晰夢を見た被験者の人数の相関図である。被験者の数が少ないので人数を増やしてもう一度実験をする必要がある。\\
\begin{figure}[htbp]
\begin{center}
\includegraphics[width=12cm]{eps/experimentDates.eps}
\caption{DreamDateの試用期間と明晰夢を見た回数の相関図}
\label{dates}
\end{center}
\end{figure}

実験を通して明らかにできたことを以下にまとめる。
\begin{itemize}
\item 被験者の思い出とより直接的な音を流すと夢を制御できる確率が高まる
\end{itemize}

また、本実験を通してDreamDateがMnemonic Induction of Lucid Dreams (The MILD Technique)に比べてユーザの負担の少ない夢制御システムであるということが示唆された。The MILD Techniqueを通して明晰夢の習得をするのには3ヶ月から1年間かかると言われている\cite{LaBerge}。第1章でも述べたがThe MILD Techniqueにはリアリティーチェックなど日常生活において多くの習慣をつけなければならない。そのため明晰夢を体験したいという気持ちがよほど強いくないと習得するのは難しい。一方でDreamDateを使った場合ユーザは音楽の登録をし、寝る前にアプリを起動して、スマートフォンを夜通し充電し、起床後に夢日記を記入するなど多少こなさなければならないタスクがあるが、The MILD Techniqueに比べると格段に負担が減る。加えて15日間の使用で7人中全員が、見たいと思っていた夢に近い内容の体験をすることができた。多い人で 7/10の確率で音と関連する夢をみて、少ない人で1/10の確率で音と関連する夢をみた。結果に個人差はあったが、MILDのトレーニングを経た人でも毎日明晰夢を見れるということではなくその日の体調や心境に左右されるという。MILDを提唱するDenysでも4回明晰夢を体験できれば多い方であると述べている\cite{LaBerge}。よってユーザに負担の減らして、MILDと同じもしくはそれ以上の効果が見込めることが分かった。

以下の要素がDreamDateの効果に個人差を引き起こしているか否か関しては明らかにすることができなかったため、より多くの被験者を集めて再度実験を行う必要がある。
\begin{itemize}
\item ユーザの年齢
\item ユーザがその夢を見たいと思う気持ちの強さ
\item ユーザのDreamDateの継続使用日数
\end{itemize}

\subsection{DreamDateの夢を制御できる確率を高めるためにできること}
DreamDateには思い出に直結する音を登録する:\\
 本研究を通して、DreamDateで体験できる仮想体験には限界があるということがわかった。人が夢を見るのは過去の記憶を整理するためである。そのためDreamDateでは思い出にない体験や情景を夢で再現することは困難である。実際に予備実験1では海の夢を見た被験者Aと被験者Cはどちらも過去に海に行った際の思い出を夢見たと答えた。一方で海にしばらく行っていない被験者Bは一度も海の夢を見なかった。また予備実験後に行った実験でも、自ら作曲した歌、演奏した曲、ダンスをした曲などの直接的な記憶にを連想させる音楽を流した被験者は10/30の確率、過去の思い出に関連した音楽を流した被験者は3/30の確率で明晰夢を体験したのに対し、自分は体験していないが映画などを通して間接的に体験した記憶にまつわる音楽を流した被験者は7/30の確率でしか夢を見ることができなかった。この結果は被験者の記憶と関連性の高い音を流すと効果が出やすいということが示唆する。よって明晰夢システムとしてDreamDateで選ぶべき音は思い出に直結する音である。\\

起床時の生活習慣を変える:\\
 全てのユーザが思い出とうまく連携した音を考え付くわけではない。実際に被験者 1 は音楽にあまり関心がなく、特に思い出に残る音・音楽がなかった。そこで最後に行った実験では被験者1に日常生活の中でコーヒーを飲む時に必ず特定の音楽を聴く習慣をつけてもらうことにした。すると図\ref{experiment3}でもあるように実験の最終日でその夢を見ることができた。Ivan Pavlovは条件反射という考え方を提唱している\cite{pavlov}。Pavlovは実験の中で犬に餌を与える前に決まってサイレンを鳴らし続けた結果、サイレンを鳴らすと犬の唾液量が増える現象が起きたと説明している。犬がサイレンを聞くと餌のことを反射的に考えたためだというのが通説である。被験者1は最後の夜に一度夢を見ただけのため、被験者の人数を増やしてさらに精度を上げた実験をする必要があるが、レム睡眠中に音楽が鳴ったことで被験者1に条件反射が働きコーヒーを飲む夢を見た可能性があると考えれる。もし仮説が正しければ、旅行をするときに特定の曲を意識的に繰り返し聴くことで旅行が終わった後もDreamDateでその音を流し旅行での想い出を夢で再生することが可能になる。但し、ある夢を見るために生活のあり方を変えるのは相当高いモチベーションを持ったユーザに限られるだろう。\\

起きる直前のレム睡眠時に音を流す:\\
 レム睡眠中脳は覚醒していて起きやすい状態であるため、音が鳴るとユーザを起こしてしまう可能性が高まる\cite{remNonRem}。そこで予備実験の後に行った実験では被験者を起こさないで明晰夢を促す方法を検証するために音を流すタイミング別に実験結果を検証した。するとレム睡眠中ずっと音を流すのに比べて最期のレム睡眠時にのみ音を流した場合の方が明晰夢を体験できる確率が上がったことが図\ref{result}から読み取れる。この結果からユーザの睡眠を極力害さないためには、レム睡眠中ずっとではなく起きる直前のレム睡眠のタイミングに音を流すが最適であるということが分かった。

\subsection{DreamDate によって HMD よりでユーザ個別の経験に関連する現実世界に近いコンテンツを体験できるか}
DreamDateが提供できる仮想体験の限界:\\
 DreamDateによって全ての被験者が思い出に関連した明晰夢を体験することができた。しかし体験できる夢の内容は限定されている。第2章でオンラインアンケートを行った際に空を飛ぶ夢、サイエンス・フィクションを連想させる夢、アイドルと遊ぶ夢、卒業論文の進め方を教えてくれる夢などそれまでに体験したことのない夢を見たいと答えた人がいたが、そのような夢を見るにはHMDを使った方が適切である。一方でDreamDateで体験可能なのは被験者の思い出に由来した夢である。夢を見ている際、脳は記憶の整理を行っている\cite{Zhang}。DreamDateでは脳の習性を利用して思い出と関連性のある音を流すことで、明晰夢を促す手法をとった。実験では 7人の被験者全員が明晰夢を体験することができた。予備実験2で被験者Cは希望であった遠距離恋愛中の交際相手とデートしたいという要望を実現させることができ、起床後もその体験を忘れずに覚えていた。そして物理的制約を超えて交際相手から手紙が届く夢やデートしている夢を見ることで幸せな気持ちを味わうことができた。DreamDateはVRコンテンツを制作するスキルや時間のないユーザでも気軽に始められるこという利点はあることが分かった。

明晰夢の副作用:\\
 被験者Cは明晰夢を見て起床した後には現実と夢は違うということに気づき、残念に思う気持ちを味わってしまったとも答えた。明晰夢の副作用についてはDenysがMnemonic Induction of Lucid Dreams (The MILD Technique) で過度に明晰夢を行うと、夢に依存して現実逃避したい気持ちにかられたり睡眠後疲れが取れない現象が起きると説明がある\cite{LaBerge}。よってDreamDateを使うユーザにはあらかじめその現象が起きる可能性があることを了承した上で提供する必要性がある。	% WWW視覚化
\chapter{結論}
\label{chap:coding}

本章では本研究の総括を行う。
\section{本研究の総括}
 本研究では明晰夢を実現できる新しい手法として、レム睡眠・ノ ンレム睡眠かを観測し起きる直前のレム睡眠中にユーザが望む体験に関連のある音を流すiOSスマー トフォンアプリケーションDreameDateを開発した。睡眠中夢を見ている際に記憶の整理をしている脳の習性を利用して、DreamDateでは思い出と関連性のある音刺激を提示した。本研究では以下の点を検証することを目的として、7人の被験者が合計15日間DreamDate使用する実験を行った。
\begin{itemize}
\item 睡眠中のユーザを音で刺激することでどのくらいの確率で明晰夢を促せるのか
\item DreamDateによってMILDに比べてより負担のかからないユーザ体験を提供できるのか
\item DreamDateによってHMDよりもオリジナルで現実世界に近いコンテンツを体験できるか
\end{itemize}

結果、全員 1 回以上関連する夢を見ることができた。予備実験1と本実験の結果から10人中 8人が、音で刺激を与えた夜の方が与えなかった夜に対して明晰夢を見る確率が高かったことから、音が夢への影響を持っていることが示唆される。DreamDateは高価なデバイスを購入をせずに既存のスマートフォンにアプリケーションをインストールするだけで使用開始できるツールを構築することができる。DreamDateは音や画像の登録、夢日記の記入などの作業を必要とするが、Mnemonic Induction of Lucid Dreams (The MILD Technique)に比べて負担がかからない\cite{LaBerge} 。DreamDateはHead Mounted Display(HMD)によるシュミレーションでは体験できないようなユニーク且つ現実的(ユーザの生活の中で実在する人物や空間、思い出に由来した体験)な仮想の体験を一部の被験者が示された。その効果に個人差が見られたが、DreamDateの夢制御システムの有効性が示された。さらにDreamDateの有効性を上げる要素として、

\begin{itemize}
\item 被験者の思い出とより直接的な音を流す
\item 起きる直前のレム睡眠時に音を流す
\end{itemize}

があることが分かった。DreamDateはユーザ自身の思い出と関連性のある夢を見たいというときは効果を表したが、今後はコンテンツの多様化を目指して音声編集システムの開発や音以外の刺激について検討する必要がある。本研究の実験は全体的に被験者の数が少ないため説得力のあるデータを収集できたとはいえないが、実験を通して明晰夢制御システムの開発にあたって意識するべき様々な要素を発見することができた。

\section{今後の展望}
DreamDateにはまだ改善するべき課題を多く残しており、以下に今後の課題を以下に述べる。

\subsection{スマートフォンアプリの課題}
DreamDateの機能改善\\
DreamDateは実験用に製作したアプリであるため、App Storeで一般のユーザに公開するためには機能面での再検討及び改善が望まれる。機能面ではまずユーザ自身が音楽の登録をできるようにする。次に睡眠中、節電のためにディスプレーをOFFにする。DreamDateは一晩中加速度によりモニタリングをする必要がある。現時点ではディスプレーがONになっているので、無駄に電力を消費してしまっている。スマートフォンに備わっている光センサーを利用して、ユーザがスクリーンを伏せたらディスプレーをシャットダウンさせるシステムにする必要がある。\\

DreamDateユーザ体験の改善\\
現時点ではユーザは起床後すぐにテキストを入力する仕組みになっているが、音声録音にすることで負担を減らすべきである。また寝る前に他のアプリを使用できない、一晩中充電をしなければならない、睡眠中に音によって起こされてしまうなどの問題を解決してユーザ体験の改善をする必要がある。最後に夢というのはプライベートな情報なので夢日記などのプライバシー強化のための仕組みも必要になるだろう。\\

\subsection{明晰夢を制御することの課題}
より正確且つ大規模な実験\\
本研究での予備実験1,2は筆者自身を含めて家族に協力してもらう結果になった。被験者が家族であると信ぴょう性が低いとみなされるため、被験者を再び検討して実験をやり直す必要性がある。また睡眠は被験者の寝ている空間、夢を覚えているか否かの体質、その日の行動や体調が実験の結果に大きく影響を与える。そのため被験者の数をできるだけ多く用意する必要があったが、最後の実験でも被験者を7人しか集めることができなかった。当初目的としていた信頼性高いデータを集めるために、明晰夢に意欲的な10人以上の被験者に50日以上の実験に関わってもらうことが望ましいと考える。同時にAppStoreにDreamDateを登録しより多くの人に使ってもらうことを目指す。\\

コンテンツの多様化\\
2章の明晰夢で体験したい内容について調査をして、LOVE タイプ、癒しタイプ、元気欲しいタイプ、アドベンチャータイプ、ストーリータイプや、ビジネスタイプなど様々な要望があるのに対してDreamDateでは特定の音と強く潜在的に思い出のエピソードと結びついている記憶しか夢で再現することができなかった。予備実験2で「音声」「曲」「自然音」の3つがそれぞれ与える影響を比較した際に交際相手に名前を呼ばれたり、語りかけられている音声を流した夜は音刺激により起こされて睡眠を害されていた。その後睡眠中のインプットは曲や波や森林などの自然音が好ましいと判断しその後のDreamDateの実験を重ねたが、音声でも語りかけ口調ではなく説明口調や歌声ならユーザが起こされない可能性がある。予備実験2で被験者Cは名前を呼ばれた際に自分が当事者であると直感的に脳が反応し、その問いに答えなければならないという気持ちになり起きてしまった。もし睡眠中に流す音声がユーザの名前を含んでおらず、語りかけ口調ではない場合は起こされない可能性がある。そこでDreamDateの機能として、音声と音楽のリミックスなどを制作する音楽編集システムを加えれば睡眠を害さずに好きな人の夢を見れる音声を制作できる。また本研究では睡眠中に与えるのは聴覚の刺激のみであったが、温度、湿度、振動や体制などによる刺激も試す実験を行う必要がある。最後に4章でも条件反射の可能性について述べたが、生活の中で特定の行動をするときに音楽を聴く習慣をつけることでどれだけ夢に影響を与えることができるかをより多くの被験者を集めて実験行うべきである。\\

長期的利用が被験者に及ぼす影響を調べる\\
長期的な実験における被験者の生活への影響を考慮するを必要があるだろう。今回は長くて15日間の実験となったが、長期的使用によって睡眠に支障が起きないかを調べる必要がある。睡眠は生物が体を休めるために必要不可欠な行為である。今回DreamDateを使った被験者が睡眠中に起こされてしまったという意見が多数でた。音による睡眠の悪影響について医療の専門家に確認する必要性がある。また明晰夢は経験を重ねるほど、成功率が上がると言われている\cite{LaBerge}。2ヶ月程実験を続けて効果に変化が出るか実験をする必要がある。一方でユーザが自由自在に夢を操作できるようになった場合、ユーザがどのような心境になるのかを調べる必要性がある。被験者Cは予備実験を通して夢から覚めて、現実に戻った際に切ない気持ちになってしまった。明晰夢が原因で現実を受け止められれなくなり、結果的にユーザが悲しむ事態に陥ってしまわないか確かめる必要がある。\\

医療の現場においての有効性を考える\\
本研究の内容が医療の分野で活用されれば、日々失われる記憶の修復ができたり、認知症患者が忘れたくない人や事柄をいつまでも覚えていられるようになる可能性がある。近年ストレスから鬱病や不眠症の悩み抱える社会人が増えているが、DreamDateが夢の中で喜びを与えることで多くの人々を救うことができるかもしれない。\\	% 開発手法
\chapter{実験方法とその結果}
\label{chap:ledoxea}

本章では、開発したiOSアプリ TokimekiDream の実験方法、内容、結果の詳細について説明する。

\section{実験方法}
予備実験:下記の被験者にまずどのような音楽が異境を与えやすいのか、どのタイミングで流すと影響が出やすいのかを調べるために実験を重ねた。

\subsection{実験結果}
夢は音に影響を与える。音を流すタイミングは起床直前のREM睡眠が一番効果的であった。そして音量は1〜3がちょうど良い。それ以上だと起きてしまう。iPhoneの位置は枕の横の方が良い。枕の下だとiPhoneが熱を発し流だけではなく、脳に音楽が響き渡り、目が覚めやすい。最後に音の種類は声、特に語りかけ口調の音声だとユーザーが起きてしまう確率が高かった。また音声はユーザーの思い出に由来している音楽の方が、影響を与えやすいということがわかった。これらの実験結果を踏まえて本格的な実験をした。それについて次に述べる。

\subsection{インタビュー}
被験者の4人は流す音声を決めてもらうために、インタビューをしてどのような趣味を持っていて、どのような思い出の夢を見たいのかを質問した。

\subsection{実験}
また、夢日記を実験の1週間前から記録してもらった。人は見た夢をすぐ忘れてしまう習性があるため、夢日記を夢を記憶できる体質になってもらう。次に
DreamScapeを20日間使い続けてもらい、REM睡眠中に流れる音声が夢に影響を与えることに成功するかいなかを実験した。実験は音楽なしと音楽ありを交互にして、時には寝る前に思い出に関する画像をみたり、思い出について10分間考えならが寝る意識をしてもらった。

\section{実験結果}
音楽なしのときは・・・(まだ実験中)
音声ありのとき・・・(まだ実験中)
音声・寝る前に画像をみる・寝る前の瞑想ありのとき・・・(まだ実験中)
継続的に使い続けるほど利用に対する希望するか否か(まだ実験中)
	% Androidアプリ「LEDOXEA」
\chapter{結論}
\label{chap:result}

本章では、本研究の総括を行う。本研究では睡眠中に思い出に関連した音を流すことでその音に関連した夢を見ることを促進するシステム、DreamScapeを提案、施策した。明晰夢をスマートフォンアプリによって誘発するためのある程度の成果と、今後の課題や方針を得られたと考える。それらをほんん研究の総括に示す。

\section{本研究の総括}
本研究では睡眠中に思い出に関連した音を流すことでその音に関連した夢を見ることを促進するシステム、DreamScapeを提案、施策した。DreamScapeによってユーザは思い出を夢で再生することができるようになり、拡張現実を体験できるようになれば、物理的に会うことのできない人と会話をしたり、過去の思い出でもう一度過ごすことで睡眠をより楽しむことができるなど、これまでにない新しい睡眠のスタイルの実現となる。
 拡張現実や睡眠に関する調査から睡眠中の夢をコントロールすることにはニーズは充分あると確信し、より多くの人たちが簡単にアクセスできるようにスマートフォンアプリ、DreamScapeを開発をした。携帯に備わっている加速度センサーでREM睡眠(夢を見ているタイミング)を観測し、そのタイミングでユーザー記憶を思い出させる音を流し、ユーザーの夢を刺激、誘導すことを目的としたシステムである。
 4人の被験者に合計20日間DreamScapeを睡眠中に使用してもらった結果、最後の7日間に関しては65\%の確率で音に連想する夢をみることに成功した。明晰夢を実現するために効果的な音楽の種類や音楽を流すタイミングなどを知ることができた。またDreamScapeの導入により、睡眠がより楽しいものになると見込まれた。


\section{今後の展開}
今回の解決すべき命題は、
\begin{itemize}
\item アプリの製作
\item 音による睡眠の悪影響につちえ医療の専門家に確認
\item 実験の制度をあげて大人数での実験を行う
\item 夢を見るために現実を加工することも検討
\item 医療の現場においての有効性を考える
\end{itemize}
の5点であった。

\subsection{アプリの製作}
実験において、被験者からもっと簡単に音楽の登録をできるようにしたいという意見が出た。ここからDreamScapeの機能面での再検討及び改善が望まれる。

\subsection{音による睡眠の悪影響につちえ医療の専門家に確認}
睡眠中に起こされてしまったという意見がでた。音による睡眠の悪影響について医療の専門家に確認する必要性がある。

\subsection{実験の精度をあげて大人数での実験を行う}
実験の精度をあげて音楽を流すタイミング、期間、音の種類をについてより細かい実験を行う。そのためにもアプリをAppStoreに登録することでより多くの人たちに使ってもらう。

\subsection{夢を見るために現実を加工することも検討}
DreamScapeの最大の難点は思い出に関連する音楽を見つけなければならないことだ。人によっては音楽をあまり聞かない人もいる。そこで生活のあり方を変えてみるのだ。例えばコーヒーを飲む時に必ず特定の音楽を聞くことで、その音楽を流すことで夢が操作される可能性が高まる。しかし生活がDreamScapeを使用することで変容していくと予想されるため、それが人によって良い影響であるのかの検討も必要である。

\subsection{医療の現場においての有効性を考える}
本研究の分野が多岐にわたって進めば日々失われる記憶の修復ができたり、忘れたくない人をいつまでも覚えていられるようになる可能性がある。認知症や鬱病を抱えている患者に喜びを与えられたり、悪夢に悩まされている人々を救うことができる。
	% アンケートによる評価と考察
%\chapter{結論}
\label{chap:conclusion}
本研究では睡眠中に思い出に関連した音を流すことでその音に関連した夢を見ることを促進するシステム、DreamScapeを提案、施策した。明晰夢をスマートフォンアプリによって誘発するためのある程度の成果と、今後の課題や方針を得られた。それらを以下に示す。

\section{本研究の総括}

 本研究では睡眠中に思い出に関連した音を流すことでその音に関連した夢を見ることを促進 するシステム、DreamMemory を提案、施策した。明晰夢をスマートフォンアプリによって誘発す るためのある程度の成果と、今後の課題や方針を得られたと考える。\\
  6 人の被験者に合計 20 日間 DreamMemory を睡眠中に使用してもらった結果、最後の 7 日間に 関しては 65\%の確率で音に連想する夢をみることに成功した。よって外的刺激により夢をある程 度操作することは可能であると証明ができた。  音楽を流すのに効果的なタイミングは起きる直前の REM 睡眠である。そして年齢が若いユー ザーの方が比較的夢を操作しやすいということがわかった。音に関しては、人の声などを交える とユーザーが起こされることが分かったので、音声でなく音楽などの方が好ましい。日常的に音 楽を聞かないユーザーに対しても効果が得られた。具体的にはコーヒーを飲むたびに同じ音楽を 聴いてもらった。すると、睡眠中にその音楽を流したときにコーヒーの夢を見ることができたの である。\\
  DreamMemory によってユーザは思い出を夢で再生することができるようになり、物理的に会 うことのできない人と会話をしたり、過去の思い出でもう一度過ごすことで睡眠をより楽しむこ とができるなどこれまでにない新しい睡眠のスタイルの実現となる。


\section{今後の展開}

今回の解決すべき命題は、
\begin{itemize}
\item アプリの製作
\item 音による睡眠の悪影響についてえ医療の専門家に確認 
\item より大人数の実験を行う
\item 医療の現場においての有効性を考える
\end{itemize}
の3点であった。

\subsection{アプリの製作}
実験において被験者から自分たちで音楽の登録ができるようにしたいという意見が出た。ここから DreamMemory の機能面での再検討及び改善が望まれる。

\subsection{音による睡眠の悪影響について医療の専門家に確認}
睡眠中に起こされてしまったという意見がでた。音による睡眠の悪影響について医療の専門家に確認する必要性がある。


\subsection{夢を見るために現実を加工することも検討}
DreamScapeの最大の難点は思い出に関連する音楽を見つけなければならないことだ。人によっては音楽をあまり聞かない人もいる。そこで生活のあり方を変えてみるのだ。例えばコーヒーを飲む時に必ず特定の音楽を聞くことで、その音楽を流すことで夢が操作される可能性が高まる。しかし生活がDreamScapeを使用することで変容していくと予想されるため、それが人によって良い影響であるのかの検討も必要である。

\subsection{医療の現場でにおいての有効性を考える}
本研究の分野が多岐にわたって進めば日々失われる記憶の修復ができたり、忘れたくない人をいつまでも覚えていられるようになる可能性がある。認知症や鬱病を抱えている患者に喜びを与えられたり、悪夢に悩まされている人々を救うことができる。
本研究では光と音に焦点を当てて刺激を発生させる装置 を提案したが,睡眠は他にも数多くの刺激と関係性がある. 例えば温度や湿度,振動(タッピング)などは睡眠の良し 悪しに大きく影響を与える.今後の展望としてこれらの刺 激も与えられるようになれば,より睡眠の質を向上させる ことができると思われる.
	% まとめ


\begin{acknowledgment}
本研究を進めるにあたり、絶えず懇切丁寧なご指導を頂きました、慶應義塾環境情報学部中西泰人教授に深く感謝いたします。 また多くの方々にアンケート調査、使用実験、本研究のフィードバックにご協力いただきました。ここに深い感謝の念を表します。 最後に、慶應義塾大学中西研究室学部生の皆様に深く感謝し、謝辞といたします。 
\end{acknowledgment}
	% 謝辞。要独自コマンド、include先参照のこと
\begin{bib}[100]


% \bibitem{参照用名称}
%   著者名: 
%   \newblock 文献名,
%   \newblock 書誌情報,出版年.

\bibitem{vrtrendShiny}
\begin{flushleft}
  Shiny Mathew:
  \newblock Importance of Virtual Reality in Current World,
  \newblock International Journal of Computer Science and Mobile Computing A Monthly Journal of Computer Science and Information Technology IJCSMC, Vol. 3, Issue. 3, 3, March 2014, pg.894 \UTF{2013} 899,
\end{flushleft}

\bibitem{vrtrendSamuel}
\begin{flushleft}
  Samuel Ebersole:
  \newblock A Brief History of Virtual Reality and its Social Applications, Academia.edu,
  \newblock Jan 1, 1997,  pg.3.
\end{flushleft}

\bibitem{dream}
\begin{flushleft}
『大辞泉 第二版』小学館:
  \newblock 2012 年 11 月 2 日
 \end{flushleft}

\bibitem{saintDenys}
\begin{flushleft}
Hervey de Saint-Denys:
  \newblock Transl.: Dream and the Ways to Direct Them: Practical Observations
  \newblock 1867
  \newblock Paris: Librairie d'Amyot, \UTF{00C9}diteur, 8, Rue de la Paix.
 \end{flushleft}
 
  \bibitem{LaBerge}
\begin{flushleft}
 LaBerge Stephen and Howard Rheingold:
  \newblock Exploring the World of Lucid Dreaming,
  \newblock New York: Ballantine, 1990. Print.
 \end{flushleft}
 
  \bibitem{tsukuba}
\begin{flushleft}
   \newblock "浅い眠りのレム睡眠、記憶定着促す役割 筑波大など発表:朝日新聞デジタル." ,
  \newblock 朝日新聞デジタル. N.p., 25 Oct. 2015.
  \newblock Web. 16 Jan. 2016.
  \newblock{\it http://www.asahi.com/articles/ASHBR316YHBRULBJ004.html}.
\end{flushleft}

  \bibitem{Dement}
\begin{flushleft}
  Dement, W.C:
  \newblock  The promise of sleep,
  \newblock Delacorte Press,
  \newblock 1999,
 \end{flushleft}
 
 \bibitem{freud}
\begin{flushleft}
  Sigmund Freud:
  \newblock The Interpretation of Dreams (1900).,
  \newblock Sigmund Freud: The Interpretation of Dreams</i>. N.p., n.d. 2015.
\end{flushleft}
 
  \bibitem{Zhang}
\begin{flushleft}
 Zhang, Jie:
  \newblock Memory process and the function of sleep,
  \newblock Journal of Theoretics Retrieved March 13, 2006.
 \end{flushleft}
 
 \bibitem{iSleep}
\begin{flushleft}
  Tian Hao, Guoliang Xing, Gang Zhou:
  \newblock  iSleep: Unobtrusive Sleep Quality Monitoring using Smartphones,
  \newblock SenSys’13,
  \newblock  November 11\UTF{2013}15, 2013, Rome, Italy
\end{flushleft}

\bibitem{beddit}
\begin{flushleft}
  beddit.com:
  \newblock Beddit Sleep Monitor and the Science Behind It,
  \newblock Beddit Science Leafle HR, 29 Aug. 2014.
\end{flushleft}
  
\bibitem{takaratomi}
\begin{flushleft}
  タカラトミー:
  \newblock "好きな夢が見られる? タカラの安眠グッズ「夢見工房」." ,
  \newblock ITmedia LifeStyle. N.p., n.d.
  \newblock Web. 08 Jan. 2016.
  \newblock{\it http://www.itmedia.co.jp/lifestyle/articles/0401/14/news047.html}.
\end{flushleft}

\bibitem{dreamOn}
\begin{flushleft}
  Dream:ON:
  \newblock "Dream:ON - The App to Influence Your Dreams.",
  \newblock The App to Influence Your Dreams. N.p., n.d. Web.
  \newblock Web. 08 Jan. 2016.
  \newblock{\it http://www.dreamonapp.com/}.
\end{flushleft}

\bibitem{yumemiru}
\begin{flushleft}
  Yumemiru:
  \newblock  "見たい夢が見られるかも?|ユメミ〜ル.",
  \newblock Web. 16 Jan. 2016.
  \newblock{\it http://www.technology-miraiworks.com/yume/}.
\end{flushleft}

\bibitem{DreamDream}
\begin{flushleft}
  Dream:ON:
  \newblock "夢操作アプリ- Dream×Dreamを App Store で.",
  \newblock App Store. N.p., n.d. Web. 16 Jan. 2016.
  \newblock{\it https://itunes.apple.com/jp/app/meng-cao-zuoapuri-dream-dream/id509989267?mt=8}.
\end{flushleft}

\bibitem{remNonRem}
\begin{flushleft}
  Patrick McNamara, Patricia Johnson, Deirdre McLaren, Erica Harris, Catherine Beauharnais, Sanford Auerbach:
  \newblock "International Review of Neurobiology",
  \newblock Volume 92, 2010, Pages 69\UTF{2013}86.
\end{flushleft}

\bibitem{negaeri}
\begin{flushleft}
  Kevin Morton:
  \newblock "The Five Stages of Sleep: Characteristics of Non-REM REM.",
  \newblock Stanford Sleep Book, n.d. Web. 15 Jan. 2016.
  \newblock{\it http://www.end-your-sleep-deprivation.com/stages-of-sleep.html}.
\end{flushleft}

\bibitem{reasonToDream}
\begin{flushleft}
  野村総合研究所 金融ITイノベーション研究部:
  \newblock 夢─脳内情報整理の仕組み─,
  \newblock 「金融 IT フォーカス」編集事務局,
  \newblock  2006,
\end{flushleft}

\bibitem{forgetDreams}
\begin{flushleft}
  Dharani, Krishnagopal:
  \newblock The Biology of Thought: A Neuronal Mechanism in the Generation of Thought,
  \newblock a New Molecular Model. N.p.: n.p., n.d. Print.
\end{flushleft}

\bibitem{cortex}
\begin{flushleft}
  Trimble, M. R.:
  \newblock The Prefrontal Cortex: Anatomy, Physiology and Neuropsychology of the Frontal Lobe.,
  \newblock  British Journal Of Psychiatry (1989).
\end{flushleft}

\bibitem{omron}
\begin{flushleft}
  Omron:
  \newblock  "ねむり時間計 HSL-002C|睡眠計・ねむり時間計|商品情報 | オムロン ヘルスケア",
  \newblock オムロン ヘルスケア. N.p., n.d. Web. 15 Jan. 2016,
  \newblock{\it http://www.healthcare.omron.co.jp/product/hsl/hsl-002c.html}.
\end{flushleft}

\bibitem{neuroon}
\begin{flushleft}
  Neuroon:
  \newblock  "Neuroon - an Intelligent Sleep Mask."
  \newblock Neuroon. N.p., n.d. Web. 08 Jan. 2016.
  \newblock{\it https://neuroon.com/technology/}.
\end{flushleft}

\bibitem{beWellApp}
\begin{flushleft}
  Zhenyu Chen, Mu Lin, Fanglin Chen, Nicholas D. Lane, Giuseppe Cardone, Rui Wang, Tianxing Li, Yiqiang Chen, Tanzeem Choudhury, Andrew T. Campbell:
  \newblock  Unobtrusive Sleep Monitoring using Smartphones,
  \newblock 2013 7th International Conference on Pervasive Computing Technologies for Healthcare and Workshops, pg. 145~152,
  \newblock 2013.252148
\end{flushleft}

\bibitem{sleepSheep}
\begin{flushleft}
  長塚 麻美, 串山 久美子, 馬場 哲晃:
  \newblock  The Interface to Improve the Quality of Sleep by the Sleep Depth Determination Using the Motion Detection,
  \newblock 情報処理学会 インタラクション 2015 IPSJ Interaction 2015,
  \newblock  A16 2015/3/5,
\end{flushleft}

\bibitem{iWinks}
\begin{flushleft}
iWinks:
 \newblock  "The Ultimate Lucid Dreaming Tool." ,
  \newblock N.p., n.d. Web. 08 Jan. 2016,
  \newblock{\it https://iwinks.org/}.
\end{flushleft}

\bibitem{roseDream}
\begin{flushleft}
Carroll:
 \newblock  "Smells Influence Dreams" ,
  \newblock National Geographic. National Geographic Society, n.d. Web. 17 Jan. 2016,
  \newblock{\it http://news.nationalgeographic.com/news/2008/09/080923-smell-sleep.html}.
\end{flushleft}

\end{bib}	% 参考文献。要独自コマンド、include先参照のこと
\appendix
\chapter{付録}
実装したアプリのソースコードを掲載する。実装はswift7.0を用いて、iPhone 9.2での動作を確認している。これらソースコードパッケージの最新版はGithubの以下のURLにて公開している。 \\
https://github.com/risahiyama

\section{DreamDateプロトタイプ}
DreamDateがスマートフォンアプリで睡眠中に音を流すという形に至った背景を述べる。
\subsection{刺激提示のプロトタイピング}
人には視覚、聴覚、触覚、味覚、嗅覚を含む5つの感覚器がある。本研究ではそのうちの聴覚と嗅覚による刺激が睡眠中の夢に与える影響を実験を通して観察した。

\subsubsection{香りによる刺激}
\begin{figure}[htbp]
\begin{center}
\includegraphics[width=9cm]{eps/smell.eps}
\caption{香りによる刺激}
\label{smell}
\end{center}
\end{figure}

 ラベンダーやバラのような良い香りは睡眠に良い影響を与え、心地良い夢を見やすくするということはMichael SchredlとBoris Stuckの研究によって証明されている\cite{roseDream}。しかし香りが夢の内容に影響を与えるか否かの研究はまだ行われていない。そこでこのプロトタイプはレム睡眠の時に思い出と直結する香りを出して夢を刺激することで夢になんらかの影響を与えられるものか否かを確かめるために製作した。\\
 加速度センサーでレム睡眠を検出したらアロマランプに光がつき、5分後香りが部屋中に充満するという作りになっている。図\ref{smell}にあるのはそのプロトタイプの写真である。実験に参加したのは嗅覚が正常に機能している(風邪などを引いていない)22歳の女性3名。被験者1には交際相手が部屋で使っているアロマとコーヒー豆の香りで刺激した。被験者2と被験者3はコーヒー豆の香りで刺激した。その香りをたくとすぐに過去の思い出と直感的に繋がる香りをあえて選び、アロマライトは被験者の頭のすぐ横に置いた。\\
 2015年の1月に10日間の実験を行った。香りありの夜、香りなしの夜を5日間ずつ交互に繰り返した。その結果を図\ref{smellExperiment}に示す。

\subsubsection{音による刺激}
 同じ被験者に今度は香りではなく音によるインプットをしてもらった。レム睡眠中に海の音や交際相手と一緒に聞いた音楽を流した。すると音によっては起こされてしまったり、被験者によっては全く影響が出ないという結果になった。しかし、図\ref{smellExperiment}が示すように、香りのインプットでは影響が全くなかったのに比べ、音のインプットは被験者1、被験者2ともに音による刺激で1日だけ夢を観たことがわかった。

\begin{figure}[htbp]
\begin{center}
\includegraphics[width=15cm]{eps/smellExperiment.eps}
\caption{香りによる刺激の実験結果}
\label{smellExperiment}
\end{center}
\end{figure}

\subsection{睡眠観測のプロトタイピング}
ユーザの睡眠深度のモニタリング方法はいくつかある。それぞれの方法をユーザビリティと機能性の2つの観点から、実験を通して分析する。

\subsubsection{脳波センサーによる観測}
 このプロトタイプではNeuroSkyのThinkGear ASICモジュールという脳波センサーを使用した。Theta波が4〜7.2HzかつDelta波が0.5〜4Hzである時をレム睡眠中であるとし自らが実験台となり装着して寝てみた。しかしこの手法は頭を締め付けられる感覚があり、且つ汗をかいてしまうのでユーザに負担がかかる。寝心地を損ねてしまうということがわかりセンサーの体と離す別の方法を試すことにした。
\begin{figure}[htbp]
\begin{center}
\includegraphics[width=10cm]{eps/brainWave.eps}
\caption{脳波センサーによるセンシングのプログラムと睡眠ステージと脳波の数値}
\label{brainWave}
\end{center}
\end{figure}

\subsubsection{心拍センサーによる観測}
 次に市販で売られている心拍センサーを追懐睡眠中の心拍数を観測することでレム睡眠を検出できるかどうかの実験をした。しかし寝ているときに指にセンサーを装着するのは発汗のを引き起こし、ユーザ体験の視点から非常に好ましくないということがわかりまたしても別の方法を試すことにした。

\begin{figure}[htbp]
\begin{center}
\includegraphics[width=10cm]{eps/heart.eps}
\caption{心拍センサーによるセンシング}
\label{heart}
\end{center}
\end{figure}

\subsubsection{kinectによる観測}
 このプロトタイプはKinectを使用して、ユーザの寝返りを検知して音楽を流すシステムである。図\ref{kinect}のようにkinectを天井に設置する。ウェラブルセンサーではないためユーザには負担がかからない。但し布団をかぶってしまうとkinectによる骨格トラッキングは難しい。そのためOpenCVのライブラリを利用して、画像処理を行った。\\
 寝返り判定の正確性を確かめるために、実際にベッドの上で寝返りを打ったとき音が鳴るかを試したところ、開発したプログラミングではノイズが多く出て誤作動が起きてしまうのでkinectを使うのは適切ではないと判断した。しかしプログラミングの能力が高い人により開発されれば、kinectによるトラッキングの精度もあげられるはずである。ただし、デバイス自体の価格が高いのと、取り付けに労力が必要とされることと、ポータブルではないため旅先では使えないという点で、本研究では好ましくないとした。

\begin{figure}[htbp]
\begin{center}
\includegraphics[width=15cm]{eps/kinect.eps}
\caption{kinectによるセンシング}
\label{kinect}
\end{center}
\end{figure}

\subsubsection{スマートフォンの加速度センサによる観測}
 最終的に多くの人々が既に使用していて、ユーザビリティーの視点から見てもっとも負担のかからないスマートフォンアプリケーションによるセンシングに試みた。スマートフォンで計測すときはウェラブルではないため身軽であるし、持ち運びが簡単なので旅中も使える。スマートフォンアプリケーションによるセンシング方法とその正確性については5章で述べる。

\section{オンラインアンケートの質問項目}
 第2章でオンラインアンケートの結果を述べているが下記がそのアンケートの質問項目である。
\begin{figure}[htbp]
\begin{center}
\includegraphics[width=8cm]{eps/interviewQuestion.eps}
\caption{オンラインアンケートの質問項目}
\label{interviewQuestion}
\end{center}
\end{figure}		% 付録

\end{document}
