% ■ 概要の出力 ■
%		begin{jabstract}〜end{jabstract}	:日本語の概要
%		begin{eabstract}〜end{eabstract}	:英語の概要
%		※ 不要ならばコマンドごと消せば出力されない。

% 日本語の概要
\begin{jabstract}
 近年Virtual Reality(VR)に関する研究開発が盛んに行われている。大規模な3次元ディスプレイを用いたVRコンテンツは制作コストが高いという問題があったが、Head Mounted Display(HMD)やスマートフォンを使ったVRのシステムが一般に普及しつつある。しかしながら、仮想現実感をもたらす3次元のコンテンツを作成するには技術やコストが高く、ユーザが望むコンテンツを気軽に作れるようにはなっていない。\\
 そこで本研究では、新たな種類の仮想現実を体験するための新しい手法として、明晰夢(睡眠中にみる夢のうち、自分で夢を見ていることを自覚していながら見ている夢)に着目し、明晰夢を見ていると推測できる状況でユーザが望む体験に関連のある音を流すという手法を提案する。\\
 明晰夢の体験者は自らの意志で夢の方向性を度々変化させることができることが知られている。そうした人々は夢の中で現実味がありながらも現実とは異なる体験をすることが可能である。しかしながら、そのように明晰夢を見るには特殊な訓練が必要とされている。  \\
 そこで本研究では特別な訓練をしなくても明晰夢を実現できる新しい手法として、レム睡眠・ノンレム睡眠かを観測し起きる直前のレム睡眠中にユーザが望む体験に関連のある音を流すスマートフォンアプリケーションであるDreameDateを開発した。\\
 提案手法の有効性を検証すべく、DreamDateが流す音が夢の内容にどのような影響を及ぼすかを調査した。人は夢を見ている際に脳は記憶の整理をしているという研究結果があるため、被験者がかつて実際に体験したことと関連の高い音を利用した。DreamDateの基本機能である睡眠の深さを観測する機能および音再生機能に、夢日記の記録機能を加え、7人の被験者に合計15日間DreamDateを使用してもらった。その結果、7人中7人が1回以上それぞれの音と関連のある夢を見た。本研究が提案した手法により夢の内容に影響を及ぼし、新たな仮想現実体験を提供することができた。

% 本研究では仮想現実の新しいアプローチを見出すことを目的とし、睡眠中の夢を音で操作することで仮想現実を体験することができるのか否かについて調べた。\\
% 近年Virtual Reality(VR)に関する研究・開発が進んでいる。しかしVRコンテンツの制作には多額の金銭的・時間的なコストがかかりバリエーションに限界があるとされているため、仮想現実を体験するための新しい取り組みが求められる。そこで本研究では明晰夢(睡眠中にみる夢のうち、自分で夢を見ていることを自覚していながら見ている夢のこと)に注目した。明晰夢の体験者は度々夢の方向性を変化させることができると知られている。そして夢の中で現実味があり且つユニークな体験をすることが可能なのだ。しかし明晰夢を試みるには特殊な訓練が必要とされている。\\
% そこで本研究では特別な訓練をしなくても明晰夢を実現することができる新しい方法を検証した。具体的にはDreamDateというスマートフォンアプリケーションを開発し、起きる直前のREM睡眠中に音を流し、その音が夢に影響を与えるのか否かについて調べた。人は夢を見ている時脳は記憶の整理をしているという研究結果があるため、音は被験者の想い出と関連性の高い音を利用した。DreamDateに睡眠の深さを観測する機能、音楽再生機能と、夢日記機能を加えた後に合計15日間7人の被験者に使用してもらった。結果、7人中7人が1回以上連想する夢を見ることに成功した。この結果は音が夢に影響を及ぼすことができるということを示すものである。

\end{jabstract}

% 英語の概要
\begin{eabstract}
Virtual Reality became a great trend, however it takes many work to program Virtual Reality contents. Therefore alternate ways of experiencing extended reality are in huge need. People actually experience extended reality while asleep, during dreams. Therefore I focused on the idea of Lucid Dreaming. Lucid Dreaming is when people notice that they are in their dream, and have control over their dream. DreamDate is an app to direct dreams by playing music to effect their dream. I was able to prove that certain sounds have effect on people’s dream. Through out this paper, I will explain the details about the system, the effect which DreamDate had on the target users.
\end{eabstract}