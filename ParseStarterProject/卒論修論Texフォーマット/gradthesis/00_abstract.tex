% ■ 概要の出力 ■
%		begin{jabstract}〜end{jabstract}	:日本語の概要
%		begin{eabstract}〜end{eabstract}	:英語の概要
%		※ 不要ならばコマンドごと消せば出力されない。

% 日本語の概要
\begin{jabstract}
 本研究では仮想現実の新しいアプローチを見出すことを目的とし、睡眠中の夢を音で操作することで仮想現実を体験することができるのか否かについて調べた。\\
 近年Virtual Reality(VR)に関する研究・開発が進んでいる。しかしVRコンテンツの制作には多額の金銭的・時間的なコストがかかりバリエーションに限界があるとされているため、仮想現実を体験するための新しい取り組みが求められる。そこで本研究では明晰夢(睡眠中にみる夢のうち、自分で夢を見ていることを自覚していながら見ている夢のこと)に注目した。明晰夢の体験者は度々夢の方向性を変化させることができると知られている。そして夢の中で現実味があり且つユニークな体験をすることが可能なのだ。しかし明晰夢を試みるには特殊な訓練が必要とされている。\\
 そこで本研究では特別な訓練をしなくても明晰夢を実現することができる新しい方法を検証した。具体的にはDreamTravelerというスマートフォンアプリケーションを開発し、REM睡眠中(夢を見ている時)に音を流し、その音が夢に影響を与えるのか否かについて調べた。\\ 
 DreamTravelerに睡眠の深さを観測する機能、音楽再生機能、夢日記機能を加えた後に、合計7人の被験者に15日間の実験を行った。結果、最後の7日間に関しては60\%の確率で音に連想する夢をみることに成功した。この結果から、音が夢に影響を及ぼすことができるということと、ユーザーにとって音が記憶と関連性が高ければ高いほど夢に影響が高いということが示唆された。

\end{jabstract}

% 英語の概要
\begin{eabstract}
	Virtual Reality became a great trend, however it takes many work to program VR contents. Therefore alternate ways of experiencing extended reality are in huge need. People actually experience extended reality while asleep, during dreams. Therefore I focused on the idea of Lucid Dreaming. Lucid Dreaming is when people notice that they are in their dream, and have control over their dream. Dreamtraveler is an app to direct dreams by playing music to effect their dream. I was able to prove that certain sounds have effect on people's dream. Through out this paper, I will explain the details about the system, the effect which Dreamtraveler had on the target users.
\end{eabstract}