% ■ 概要の出力 ■
%		begin{jabstract}〜end{jabstract}	:日本語の概要
%		begin{eabstract}〜end{eabstract}	:英語の概要
%		※ 不要ならばコマンドごと消せば出力されない。

% 日本語の概要
\begin{jabstract}
 近年Virtual Reality(VR)に関する研究開発が盛んに行われている。大規模な3次元ディスプレイを用いたVRコンテンツは制作コストが高いという問題があったが、Head Mounted Display(HMD)やスマートフォンを使ったVRのシステムが一般に普及しつつある。しかしながら、充分な仮想現実感をもたらす3次元のコンテンツを作成するには技術やコストが必要とされ、ユーザが望むコンテンツを気軽に作れるようにはなっていない。\\
 そこで本研究では、新たな種類の仮想現実を体験するための新しい手法として、明晰夢(睡眠中に見る夢のうち自分で夢を見ていることを自覚していながら見ている夢)に着目し、明晰夢を見ていると推測できる状況でユーザが望む体験に関連のある音を流すという手法を提案する。\\
 明晰夢の体験者は自らの意志で夢の方向性を度々変化させることができることが知られている。そうした人々は夢の中で現実味がありながらも現実とは異なる体験をすることが可能である。しかしながら、そのように明晰夢を見るには特殊な訓練が必要とされている。 \\
 そこで本研究では特別な訓練をしなくても明晰夢を実現できる新しい手法として、レム睡眠・ノンレム睡眠かを観測し起きる直前のレム睡眠中にユーザが望む体験に関連のある音を流すスマートフォンアプリケーションであるDreameDateを開発した。これによりユーザが夢で会いたいと望む人や過ごしたいと思う場所に関連した夢を見る確率が上がる。\\
 提案手法の有効性を検証すべくDreamDateが流す音が夢の内容にどのような影響を及ぼすかを調査した。人は夢を見ている際に脳は記憶の整理をしているという研究結果があるため、被験者がかつて実際に体験したことと関連の高い音を利用した。DreamDateの基本機能である睡眠の深さを観測する機能および音再生機能に、夢日記の記録機能を加え、7人の被験者に合計15日間DreamDateを使用してもらった。その結果7人中7人が1回以上それぞれの音と関連のある夢を見た。実験結果からユーザの記憶と睡眠中に流す音の関係性が強ければ強いほど明晰夢の起きる可能性が高いということが示唆された。

\end{jabstract}

% 英語の概要
\begin{eabstract}
Over the past few years, there has been an extensive research and development on Virtual Reality (VR). Although it used to be hard to create large scale 3 dimensional display, VR system using Head Mounted Display (HMD) and smartphones are becoming popular in general. However, to create a 3 dimensional content that results in sufficient virtual experience requires technique and cost. In this study, to experience the virtual reality of a new type, we focused on lucid dream.  A lucid dream is any dream in which one is aware that one is dreaming and therefore has control over the scenario of the dream.However, special training is required to experience lucid dream.  However, lucid dreaming requires extensive training called Mnemonic Induction of Lucid Dreams. Therefore, we propose DreamDate. DreamDate is a smartphone application which plays a sound related to user's memory during REM sleep. By using this App, the possibility of experiencing lucid dreaming will increase.Throughout the research, we experimented different kinds of sound to see which has effect on influencing dreams. We found out that sounds that relate to user's unique experience are more effective. As a result, 7 out of 7 participants who took part in the 15 days experiment dreamed about what was associated with the sound they selected. This suggests that the stronger the relationship between the sound and user's unique experience the more likely the users are able to have control over their dream.
\end{eabstract}