% ■ 概要の出力 ■
%		begin{jabstract}〜end{jabstract}	:日本語の概要
%		begin{eabstract}〜end{eabstract}	:英語の概要
%		※ 不要ならばコマンドごと消せば出力されない。

% 日本語の概要
\begin{jabstract}
 本研究では睡眠中に心に残る思い出に関連した音を流すことで、その音に促された夢を見ることができるかについて調べた。\\
 近年Virtual Reality(VR)に関する研究・開発が進んでいる。しかしVRコンテンツの制作には多額の金銭的・時間的なコストがかかりバリエーションに限界があるとされているため、拡張現実を体験するための新しい取り組みが求められる。そこで本研究では明晰夢(睡眠中にみる夢のうち、自分で夢を見ていることを自覚していながら見ている夢のこと)に注目した。明晰夢の体験者は度々夢の方向性を変化させることができると知られている。そして夢の中で現実味があり且つユニークな体験をすることが可能なのだ。しかし明晰夢を試みるには特殊な訓練が必要とされている。\\
 そこで本研究では外的刺激によって誰もが明晰夢をより簡単に実現することができるのかを検証するために、MemoryDreamの開発をした。MemoryDreamは睡眠時にユーザーの記憶を呼び起こす音を流すことでユーザーの夢を刺激し、夢を誘導すこと目的としたアプリケーションである。具体的にはスマートフォンに備わっている加速度センサーで体動を観測し、ユーザーが夢を見ているタイミングを観測する。5人の被験者に合計20日間MemoryDreamを睡眠中に使用してもらった結果、最後の7日間に関しては65\%の確率で音に連想する夢をみることに成功した。ただし〜〜〜関与していた。(結論)〜〜
\end{jabstract}

% 英語の概要
\begin{eabstract}
	Virtual Reality have become a great trend, however it takes many work to program VR contents. Therefore alternate ways of experiencing extended reality are in huge need. \\
 	People actually experience extended reality while asleep, during dreams. Therefore I focused on the idea of Lucid Dreaming. Lucid Dreaming is when people notice that they are in their dream, and have control over their dream.\\
 	MemoryDream is an app to direct dreams by playing music to effect their dream. I was able to prove that certain sounds have effect on people's dream. Through out this paper, I will explain the details about the system, the effect which MemoryDream had on the target users.
\end{eabstract}