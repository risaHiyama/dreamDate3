% ■ 概要の出力 ■
%		begin{jabstract}〜end{jabstract}	:日本語の概要
%		begin{eabstract}〜end{eabstract}	:英語の概要
%		※ 不要ならばコマンドごと消せば出力されない。

% 日本語の概要
\begin{jabstract}
 本研究では睡眠中に思い出に関連した音を流すことで、その音に関連した夢を見ることを促進するシステム、TokimekiDreamを提案する。\\
 近年Virtual Realityの開発が進んでいるがコンテンツの制作には長時間かかりバリエーションに限界があるとされているため、拡張現実を体験するための新しい取り組みが求められる。明晰夢(睡眠中にみる夢のうち、自分で夢を見ていることを自覚していながら見ている夢のこと)の体験者は度々夢の方向性を変化させることができると知られている。そして夢の中で現実味があり且つユニークな体験をすることが可能なのだ。\\
 TokimekiDreamはスマートフォンに備わっている加速度センサーでREM睡眠(夢を見ているタイミング)を観測し、そのタイミングでユーザー記憶を思い出させる音を流し、ユーザーの夢を刺激、誘導すことを目的としたシステムである。4人の被験者に合計20日間TokimekiDreamを睡眠中に使用してもらった結果、最後の7日間に関しては65\%の確率で音に連想する夢をみることに成功した。
\end{jabstract}

% 英語の概要
%\begin{eabstract}
%Mashup enables us to search contents like a portal site. However, there is difference to classify contents with Web2.0, use webAPI in external sites, search in the client-side and  the server-side, merge individual contents as a any hybrid content, and use REST, RSS, and Atom with converted XML. This time I developed the search system that make it possible to control the graph in the 3D space by existing proposed model(NATTO View), with these peculiarity on Android devices.
%\end{eabstract}

