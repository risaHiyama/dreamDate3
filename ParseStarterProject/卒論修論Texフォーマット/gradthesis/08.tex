\chapter{結論}
\label{chap:conclusion}
本研究では睡眠中に思い出に関連した音を流すことでその音に関連した夢を見ることを促進するシステム、DreamScapeを提案、施策した。明晰夢をスマートフォンアプリによって誘発するためのある程度の成果と、今後の課題や方針を得られた。それらを以下に示す。

\section{本研究の総括}

 本研究では睡眠中に思い出に関連した音を流すことでその音に関連した夢を見ることを促進 するシステム、DreamMemory を提案、施策した。明晰夢をスマートフォンアプリによって誘発す るためのある程度の成果と、今後の課題や方針を得られたと考える。\\
  6 人の被験者に合計 20 日間 DreamMemory を睡眠中に使用してもらった結果、最後の 7 日間に 関しては 65\%の確率で音に連想する夢をみることに成功した。よって外的刺激により夢をある程 度操作することは可能であると証明ができた。  音楽を流すのに効果的なタイミングは起きる直前の REM 睡眠である。そして年齢が若いユー ザーの方が比較的夢を操作しやすいということがわかった。音に関しては、人の声などを交える とユーザーが起こされることが分かったので、音声でなく音楽などの方が好ましい。日常的に音 楽を聞かないユーザーに対しても効果が得られた。具体的にはコーヒーを飲むたびに同じ音楽を 聴いてもらった。すると、睡眠中にその音楽を流したときにコーヒーの夢を見ることができたの である。\\
  DreamMemory によってユーザは思い出を夢で再生することができるようになり、物理的に会 うことのできない人と会話をしたり、過去の思い出でもう一度過ごすことで睡眠をより楽しむこ とができるなどこれまでにない新しい睡眠のスタイルの実現となる。


\section{今後の展開}

今回の解決すべき命題は、
\begin{itemize}
\item アプリの製作
\item 音による睡眠の悪影響についてえ医療の専門家に確認 
\item より大人数の実験を行う
\item 医療の現場においての有効性を考える
\end{itemize}
の3点であった。

\subsection{アプリの製作}
実験において被験者から自分たちで音楽の登録ができるようにしたいという意見が出た。ここから DreamMemory の機能面での再検討及び改善が望まれる。

\subsection{音による睡眠の悪影響について医療の専門家に確認}
睡眠中に起こされてしまったという意見がでた。音による睡眠の悪影響について医療の専門家に確認する必要性がある。


\subsection{夢を見るために現実を加工することも検討}
DreamScapeの最大の難点は思い出に関連する音楽を見つけなければならないことだ。人によっては音楽をあまり聞かない人もいる。そこで生活のあり方を変えてみるのだ。例えばコーヒーを飲む時に必ず特定の音楽を聞くことで、その音楽を流すことで夢が操作される可能性が高まる。しかし生活がDreamScapeを使用することで変容していくと予想されるため、それが人によって良い影響であるのかの検討も必要である。

\subsection{医療の現場でにおいての有効性を考える}
本研究の分野が多岐にわたって進めば日々失われる記憶の修復ができたり、忘れたくない人をいつまでも覚えていられるようになる可能性がある。認知症や鬱病を抱えている患者に喜びを与えられたり、悪夢に悩まされている人々を救うことができる。
本研究では光と音に焦点を当てて刺激を発生させる装置 を提案したが,睡眠は他にも数多くの刺激と関係性がある. 例えば温度や湿度,振動(タッピング)などは睡眠の良し 悪しに大きく影響を与える.今後の展望としてこれらの刺 激も与えられるようになれば,より睡眠の質を向上させる ことができると思われる.
