\chapter{序論}
\label{chap:introduction}

本章では本研究の背景、それを踏まえた上での研究の目的、そして論文の構成について述べる。

\section{研究の背景}
VRとはコンピューターによりシュミレーショ ンされた仮想現実がユーザの動きにインタラクティブに反応することをいう。Virtual Reality(VR)の起源は1956年にMorton Heiligによって開発されたSensoramaに遡る。Sensoramaは視野、音、振動や香りの刺激による仮想現実体験マシンである\cite{sensorama}。日常生活を抜け出して場所や時間の制約を超えて架空の現実を体験することは長い間人類の夢であった\cite{verge}。しかし技術的制約や制作コストが高いという問題があり一般に普及は難しかった。\\
 2014年にFacebookがHead Mounted Display(HMD)を開発したOCULUSを買収してからVRの市場は一気に広がった\cite{vrtrendShiny}。そして今は世界中の企業やクリエイターにより3次元のディスプレーシュミレーション空間のが次々に生み出されている\cite{vrtrendSamuel}。OCULUSは目を完全に覆う没入型の装置で、頭部の動きを3次元トラッキングしてインタラクティブでリアルな仮想現実が体験できるようになっている。そしてOCULUSはtouchというコントローラーによりさらに直感的な操作が可能になった\cite{touch}。HTC Viveは70個のセンサーを備え、顔の動きを正確に360度トラッキングできる\cite{vive}。トラッキングの精度は高いがどちらも5万円以上で高額になっている。そこで大衆向けに安価な価格で開発され発売されるようになったのが、Samsung's Gear VR headsetである\cite{samsung}。セッティングも一般ユーザのために簡単になった。他にもダンボールで作られているGoogleのCardBoardがある。こちらはiPhoneを3Dディスプレーとして使うため2000円で購入できる。Cardboard用のアプリの数も増えており、販売後1ヶ月で1万人が使うようになった\cite{cardboard}。ところがスマートフォンによるVR体験のほとんどはコンサートやローラーコースター、美術館の体験などが多くインタラクティブなものは少ない。\\
 しかしながらHMDには幾つかの技術的制約があるとOculous VR共同創立者のNate Mitchellが述べている\cite{oculus}。例えば一人称の仮想空間の場合体験者に違和感を感じさせないために身長をユーザと一致させることで目線を合わせたり、自身の手や足などを拡張空間の中でも作り上げるために全身のトラッキングシステムを考えなかればならない。また物体や空間をよりリアルに見せるために影、テキスタイル、動き、重力や音を正確に表現しなければならならない\cite{vrtrendShiny}。そしてユーザが乗り物酔いを防ぐには1秒に75フレームのスピードでレンダリングを行わなければならい\cite{HMDifficulties}。VRコンテンツ制作には声優、デザイナー、アーティストやエンジニアが必要になる。UNITYなどのゲームエンジンなどを使うことでそれらの作業は比較的楽になったが、HMDによる仮想空間は多数向けのコンテンツに留まっている。例えば物理的壁を越えて遠くにいる人と会いたい、時間的壁を越えて過去に旅行をした時の思い出を再び体験したい、自ら好きな映画の登場人物となって刺激を感じてみたいとある個人が思っても、その人のモデリングデータをまず入力しなければならない。エンジニアリングの知識のないユーザにとってそういったパーソナルなコンテンツを作成することは困難である。\\
 そこで本研究では新たな種類の仮想現実を体験するための新しい手法として、睡眠中にみる夢のうち自分で夢を見ていることを自覚していながら見ている夢である明晰夢に着目した。意識することは少ないが人は毎晩睡眠中に人々は仮想現実を体験している。辞書 『大辞泉 第二版』によると夢とは「睡眠中にあたかも現実の経験であるかのように感じる一連の観念や心像のこと」\cite{dream}と書かれている。\\
 明晰夢は1867 年に Denys により研究が行われて以来\cite{saintDenys}、心理学者や哲学者の間で研究が進められてきた。明晰夢の経験者は夢の状況を自分の思い通りに変化させられる、言い換えれば仮想現実を作り出し体験することができると言われてきた。1987 年にスタンフォード大学の LaBergeがThe Lucidity Institute を設立し、 明晰夢を体験するためのステップをMnemonic Induction of Lucid Dreams (The MILD Technique) と命名した\cite{LaBerge}。しかしThe MILD Techniqueを習得するには特殊な訓練が必要である。例えば就寝してから5時間たつときにアラームをかけて一度起きて、明晰夢のことを念じながらもう一度寝る。また明晰夢を見るためには夢を見ているか否かを自覚できる体質にならなければならない。そのために起床中も夢を見ているのか否かを確認するリアリティーチェックという習慣を付けておく必要がある。例えば夢の本数が正確であるかを確認する、口と鼻から息をしっかりしているか確認する、あるいは鏡に覗き込み自分が映るか確認する、ジャンプをして重力を感じるか試すなど、方法は個人によって様々である。他にも夢日記を欠かさず付けて普段から夢を覚えている体質になること。眠りにつく前に体制を正して深呼吸をし心を落ち着かせ、夢で見たい内容を思い浮かべ、「これから明晰夢を見るんだ」と念じならが寝るといった方法である。このようにMILDは労力が必要で誰もが気軽に始められるものとは言い難い。

\section{研究の目的:レム睡眠時における音刺激の提示を 用いた夢制御システム"DreamDate"}
本論文ではユーザが睡眠中の夢で金銭的コストをかけることなく仮想体験を楽めるように促す、スマートフォンアプリケーション"DreamDate"を提案し、その設計や構想、実験結果を述べる。
 DreamDateは睡眠中にユーザのレム睡眠を感知して音による刺激を与えることで夢を理想的なシチュエーションへと促す夢制御システムである。夢制御システムとしては既に株式会社タカラトミーが開発した夢見工房\cite{takaratomi}やDreamON\cite{dreamOn}などのスマートフォンアプリがある。しかしそれらの実験データが公式ウェブサイトなどでも明らかにされていない。AmazonやApp Storeのユーザレビューではアイディアの新規性については評価をされているが、夢制御システムとしての効果はユーザによって大きく個人差が出ているため、その有効性については疑問が残る。そこで本研究では実験を通してDreamDateのシステムの改善を図り、夢制御システムとしての有効性を証明する。そのために以下の3の課題を検証する必要があると考える。

\begin{itemize}
\item 睡眠中のユーザを音で刺激することでどのくらいの確率で明晰夢を促せるのか
\item DreamDateによってMILDに比べてより負担のかからないユーザ体験を提供できるのか
\item DreamDateによってHMDよりもオリジナルで現実世界に近いコンテンツを体験できるか
\end{itemize}

 人生の1/3をしめる睡眠中にみる夢を自由にコントロールできれば、金銭も時間も使わずに好きな人と好きな空間で仮想体験を楽しむことができる。これまでにない新しい睡眠のスタイルの実現に貢献したい。

\section{本論文の構成}
第\ref{chap:introduction}章では、本研究の背景と目的、そして論文の構成を述べる。第\ref{chap:webapi}章ではユーザの仮想現実や明晰夢における認識度や要求についての事前調査、睡眠中に見た夢の分析を行った。また睡眠観測や明晰夢促進という目線での先行研究や開発事例を述べたのちに、複数の 観点からDreamDateとの比較を行う。第\ref{chap:search}章ではDreamDateのシステムの概要、利用方法 について述べる。第\ref{chap:visualize}章では本研究で構築した DreamDateを用いたユーザスタディの結果と考察を述べる。第\ref{chap:coding}章では本研究の総括を行い、また今後の展望について議論する。
